%% Copernicus Publications Manuscript Preparation Template for LaTeX Submissions
%% ---------------------------------
%% This template should be used for copernicus.cls
%% The class file and some style files are bundled in the Copernicus Latex Package, which can be downloaded from the different journal webpages.
%% For further assistance please contact Copernicus Publications at: production@copernicus.org
%% https://publications.copernicus.org/for_authors/manuscript_preparation.html

%% copernicus_rticles_template (flag for rticles template detection - do not remove!)

%% Please use the following documentclass and journal abbreviations for discussion papers and final revised papers.

%% 2-column papers and discussion papers
\documentclass[, manuscript]{copernicus}



%% Journal abbreviations (please use the same for discussion papers and final revised papers)


% Advances in Geosciences (adgeo)
% Advances in Radio Science (ars)
% Advances in Science and Research (asr)
% Advances in Statistical Climatology, Meteorology and Oceanography (ascmo)
% Annales Geophysicae (angeo)
% Archives Animal Breeding (aab)
% ASTRA Proceedings (ap)
% Atmospheric Chemistry and Physics (acp)
% Atmospheric Measurement Techniques (amt)
% Biogeosciences (bg)
% Climate of the Past (cp)
% DEUQUA Special Publications (deuquasp)
% Drinking Water Engineering and Science (dwes)
% Earth Surface Dynamics (esurf)
% Earth System Dynamics (esd)
% Earth System Science Data (essd)
% E&G Quaternary Science Journal (egqsj)
% European Journal of Mineralogy (ejm)
% Fossil Record (fr)
% Geochronology (gchron)
% Geographica Helvetica (gh)
% Geoscience Communication (gc)
% Geoscientific Instrumentation, Methods and Data Systems (gi)
% Geoscientific Model Development (gmd)
% History of Geo- and Space Sciences (hgss)
% Hydrology and Earth System Sciences (hess)
% Journal of Bone and Joint Infection (jbji)
% Journal of Micropalaeontology (jm)
% Journal of Sensors and Sensor Systems (jsss)
% Magnetic Resonance (mr)
% Mechanical Sciences (ms)
% Natural Hazards and Earth System Sciences (nhess)
% Nonlinear Processes in Geophysics (npg)
% Ocean Science (os)
% Polarforschung - Journal of the German Society for Polar Research (polf)
% Primate Biology (pb)
% Proceedings of the International Association of Hydrological Sciences (piahs)
% Scientific Drilling (sd)
% SOIL (soil)
% Solid Earth (se)
% The Cryosphere (tc)
% Weather and Climate Dynamics (wcd)
% Web Ecology (we)
% Wind Energy Science (wes)


%% \usepackage commands included in the copernicus.cls:
%\usepackage[german, english]{babel}
%\usepackage{tabularx}
%\usepackage{cancel}
%\usepackage{multirow}
%\usepackage{supertabular}
%\usepackage{algorithmic}
%\usepackage{algorithm}
%\usepackage{amsthm}
%\usepackage{float}
%\usepackage{subfig}
%\usepackage{rotating}

% Pandoc citation processing

% The "Technical instructions for LaTex" by Copernicus require _not_ to insert any additional packages.
%
\usepackage{algorithmic}
\usepackage{algorithm}


\begin{document}

\title{Informing IPCC accounting of forest carbon using the global
forest carbon database (ForC v4.0)}


\Author[1]{Madison}{Williams}
\Author[1]{Valentine}{Herrmann}
\Author[2]{Rebecca}{Banbury Morgan}
\Author[3]{Ben}{Bond-Lamberty (confirm)}
\Author[4]{Susan}{Cook-Patton (confirm)}
\Author[5]{Sandro}{Federici (confirm)}
\Author[6]{Helene}{Muller-Landau}
\Author[7]{Camille}{Piponiot}
\Author[1]{Teagan}{Rogers}
\Author[5]{Valentyna}{Slivinska (confirm)}
\Author[1,6 *]{Kristina J.}{Anderson-Teixeira}


\affil[1]{Center for Conservatiton Ecology, Smithsonian Conservation
Biology Institute, Front Royal, VA, United States}
\affil[2]{}
\affil[3]{}
\affil[4]{}
\affil[5]{}
\affil[6]{Forest Global Earth Observatory, Smithsonian Tropical Research
Institute, Panama, Republic of Panama}
\affil[7]{CIRAD, Montpellier, France}

\runningtitle{IPCC forest C accounting with ForC}

\runningauthor{Williams et al.}


\correspondence{Kristina J.\ Anderson-Teixeira\ (teixeirak@si.edu)}



\received{}
\pubdiscuss{} %% only important for two-stage journals
\revised{}
\accepted{}
\published{}

%% These dates will be inserted by Copernicus Publications during the typesetting process.


\firstpage{1}

\maketitle


\begin{abstract}
The abstract goes here. It can also be on \emph{multiple lines}.
\end{abstract}


\copyrightstatement{The author's copyright for this publication is
transferred to institution/company.}


The following settings can or must be configured in the header of this
file and are bespoke for Copernicus manuscripts:

\begin{itemize}
\item
  The \texttt{journal} you are submitting to using the official
  abbreviation. You can use the function
  \texttt{rticles::copernicus\_journal\_abbreviations(name\ =\ \textquotesingle{}...\textquotesingle{})}
  to search the existing journals.
\item
  Specific sections of the manuscript:

  \begin{itemize}
  \item
    \texttt{running} with \texttt{title} and \texttt{author}
  \item
    \texttt{competinginterests}
  \item
    \texttt{copyrightstatement} (optional)
  \item
    \texttt{availability} (strongly recommended if any used), one of
    \texttt{code}, \texttt{data}, or \texttt{codedata}
  \item
    \texttt{authorcontribution}
  \item
    \texttt{disclaimer}
  \item
    \texttt{acknowledgements}
  \end{itemize}
\end{itemize}

See the defaults and examples from the skeleton and the official
Copernicus documentation for details.

\textbf{Important}: Always double-check with the official manuscript
preparation guidelines at
\url{https://publications.copernicus.org/for_authors/manuscript_preparation.html},
especially the sections ``Technical instructions for LaTeX'' and
``Manuscript composition''. Please contact Daniel Nüst,
\texttt{daniel.nuest@uni-muenster.de}, with any problems.

\introduction[Introduction]

\emph{(Importance of forests for climate change mitigation)}

\emph{(Need for good data in international carbon accounting)}

\emph{(Introduce EFDb \& forc)}

Example citation \citep{anderson-teixeira_forc_2018}

Here, we: (1) clarify C cycle terminology (2) describe mapping of ForC
to IPCC's EFDB, (3) describe updates to ForC (ForC v4.0) (4) summarize
the data in ForC that's relevant to EFDB, identifying gaps (5) provide
recommendations for improving data collection, analysis, database, and
accounting

\section{Defining carbon stocks and incremenets}

For quantifying forest role in global C cycle, we ultimately care about:
(1) C stocks --stores of C that would be released to the atmosphere upon
and use change (2) C increments -- changes in those C stocks.

\subsection{Carbon stocks}

Forest ecosystem C stocks may be parsed into pools in various ways. IPCC
parses into biomass (aboveground and belowground), dead organic matter
(dead wood and litter), and soil organic matter (Table 1). Quantifying
these requires a one-time measurement.

\newpage

\textbf{Table 1: variables with definitions and measurement methods.}
Definitions from IPCC Table 1.1. (See Table 1.1 in IPCC guidance).
(\emph{Currently adding this as a figure (generated from original draft)
because kableExtra doesn't seem to work in this template, and I can't
quickly get the template format to work. Table that we want here is
``figures\_tables/C\_pools.csv''})

\begin{figure}
\includegraphics[width=12cm]{figures_tables/C_pools_fig} \caption{Table 1}\label{fig:unnamed-chunk-1}
\end{figure}

\begin{table*}[t]
\caption{This is a start at table 1 using the template format. }
\begin{tabular}{l c c c r}
\tophline

pool & subpool & definition & major sources of estimate variation & IPCC guidance \\
\middlehline
biomass & aboveground & all biomass of living vegetation, both woody and herbaceous, above the soil & allometry, min dbh & acceptable to exclude understory  \\

\bottomhline
\end{tabular}
\belowtable{I don't know how to adjust so that it doesn't run off the page.}
\end{table*}

\subsection{Carbon increments}

C increments are defined as the change over time, in annual increments,
in each C pool. These may be estimated as the difference between C
stocks at two time points, or as the difference between inputs and
outputs to the pool (i.e., fluxes). Quantifying these requires at least
two measurements.

Fluxes are the inputs and outputs to each pool.

\textbf{Figure: schematic illustrating fluxes in and out of each pool}

\conclusions[Conclusions]

The conclusion goes here. You can modify the section name with
\texttt{\textbackslash{}conclusions{[}modified\ heading\ if\ necessary{]}}.



\codedataavailability{use this to add a statement when having data sets
and software code
available} %% use this section when having data sets and software code available

\sampleavailability{use this section when having geoscientific samples
available} %% use this section when having geoscientific samples available

\videosupplement{use this section when having video supplements
available} %% use this section when having geoscientific samples available

%%%%%%%%%%%%%%%%%%%%%%%%%%%%%%%%%%%%%%%%%%
%% optional

%%%%%%%%%%%%%%%%%%%%%%%%%%%%%%%%%%%%%%%%%%
\appendix
\section{Figures and tables in appendices}

Regarding figures and tables in appendices, the following two options
are possible depending on your general handling of figures and tables in
the manuscript environment:

\subsection{Option 1}

If you sorted all figures and tables into the sections of the text,
please also sort the appendix figures and appendix tables into the
respective appendix sections. They will be correctly named
automatically.

\subsection{Option 2}

If you put all figures after the reference list, please insert appendix
tables and figures after the normal tables and figures.

To rename them correctly to A1, A2, etc., please add the following
commands in front of them: \texttt{\textbackslash{}appendixfigures}
needs to be added in front of appendix figures
\texttt{\textbackslash{}appendixtables} needs to be added in front of
appendix tables

Please add \texttt{\textbackslash{}clearpage} between each table and/or
figure. Further guidelines on figures and tables can be found below.
\noappendix

%%%%%%%%%%%%%%%%%%%%%%%%%%%%%%%%%%%%%%%%%%
\authorcontribution{Daniel wrote the package. Josiah thought about
poterry. Markus filled in for a second author.} %% optional section

%%%%%%%%%%%%%%%%%%%%%%%%%%%%%%%%%%%%%%%%%%
\competinginterests{The authors declare no competing
interests.} %% this section is mandatory even if you declare that no competing interests are present

%%%%%%%%%%%%%%%%%%%%%%%%%%%%%%%%%%%%%%%%%%
\disclaimer{We like Copernicus.} %% optional section

%%%%%%%%%%%%%%%%%%%%%%%%%%%%%%%%%%%%%%%%%%
\begin{acknowledgements}
Thanks to the rticles contributors!
\end{acknowledgements}

%% REFERENCES
%% DN: pre-configured to BibTeX for rticles

%% The reference list is compiled as follows:
%%
%% \begin{thebibliography}{}
%%
%% \bibitem[AUTHOR(YEAR)]{LABEL1}
%% REFERENCE 1
%%
%% \bibitem[AUTHOR(YEAR)]{LABEL2}
%% REFERENCE 2
%%
%% \end{thebibliography}

%% Since the Copernicus LaTeX package includes the BibTeX style file copernicus.bst,
%% authors experienced with BibTeX only have to include the following two lines:
%%
\bibliographystyle{copernicus}
\bibliography{references.bib}
%%
%% URLs and DOIs can be entered in your BibTeX file as:
%%
%% URL = {http://www.xyz.org/~jones/idx_g.htm}
%% DOI = {10.5194/xyz}


%% LITERATURE CITATIONS
%%
%% command                        & example result
%% \citet{jones90}|               & Jones et al. (1990)
%% \citep{jones90}|               & (Jones et al., 1990)
%% \citep{jones90,jones93}|       & (Jones et al., 1990, 1993)
%% \citep[p.~32]{jones90}|        & (Jones et al., 1990, p.~32)
%% \citep[e.g.,][]{jones90}|      & (e.g., Jones et al., 1990)
%% \citep[e.g.,][p.~32]{jones90}| & (e.g., Jones et al., 1990, p.~32)
%% \citeauthor{jones90}|          & Jones et al.
%% \citeyear{jones90}|            & 1990

\end{document}
