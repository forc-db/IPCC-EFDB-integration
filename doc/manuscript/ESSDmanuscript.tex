%% Copernicus Publications Manuscript Preparation Template for LaTeX Submissions
%% ---------------------------------
%% This template should be used for copernicus.cls
%% The class file and some style files are bundled in the Copernicus Latex Package, which can be downloaded from the different journal webpages.
%% For further assistance please contact Copernicus Publications at: production@copernicus.org
%% https://publications.copernicus.org/for_authors/manuscript_preparation.html

%% copernicus_rticles_template (flag for rticles template detection - do not remove!)

%% Please use the following documentclass and journal abbreviations for discussion papers and final revised papers.

%% 2-column papers and discussion papers
\documentclass[, manuscript]{copernicus}



%% Journal abbreviations (please use the same for discussion papers and final revised papers)


% Advances in Geosciences (adgeo)
% Advances in Radio Science (ars)
% Advances in Science and Research (asr)
% Advances in Statistical Climatology, Meteorology and Oceanography (ascmo)
% Annales Geophysicae (angeo)
% Archives Animal Breeding (aab)
% ASTRA Proceedings (ap)
% Atmospheric Chemistry and Physics (acp)
% Atmospheric Measurement Techniques (amt)
% Biogeosciences (bg)
% Climate of the Past (cp)
% DEUQUA Special Publications (deuquasp)
% Drinking Water Engineering and Science (dwes)
% Earth Surface Dynamics (esurf)
% Earth System Dynamics (esd)
% Earth System Science Data (essd)
% E&G Quaternary Science Journal (egqsj)
% European Journal of Mineralogy (ejm)
% Fossil Record (fr)
% Geochronology (gchron)
% Geographica Helvetica (gh)
% Geoscience Communication (gc)
% Geoscientific Instrumentation, Methods and Data Systems (gi)
% Geoscientific Model Development (gmd)
% History of Geo- and Space Sciences (hgss)
% Hydrology and Earth System Sciences (hess)
% Journal of Bone and Joint Infection (jbji)
% Journal of Micropalaeontology (jm)
% Journal of Sensors and Sensor Systems (jsss)
% Magnetic Resonance (mr)
% Mechanical Sciences (ms)
% Natural Hazards and Earth System Sciences (nhess)
% Nonlinear Processes in Geophysics (npg)
% Ocean Science (os)
% Polarforschung - Journal of the German Society for Polar Research (polf)
% Primate Biology (pb)
% Proceedings of the International Association of Hydrological Sciences (piahs)
% Scientific Drilling (sd)
% SOIL (soil)
% Solid Earth (se)
% The Cryosphere (tc)
% Weather and Climate Dynamics (wcd)
% Web Ecology (we)
% Wind Energy Science (wes)


%% \usepackage commands included in the copernicus.cls:
%\usepackage[german, english]{babel}
%\usepackage{tabularx}
%\usepackage{cancel}
%\usepackage{multirow}
%\usepackage{supertabular}
%\usepackage{algorithmic}
%\usepackage{algorithm}
%\usepackage{amsthm}
%\usepackage{float}
%\usepackage{subfig}
%\usepackage{rotating}

% Pandoc citation processing

% The "Technical instructions for LaTex" by Copernicus require _not_ to insert any additional packages.
%
\usepackage{algorithmic}
\usepackage{algorithm}


\begin{document}

\title{Informing IPCC accounting of forest carbon using the global
forest carbon database (ForC v4.0)}


\Author[1]{Madison}{Williams}
\Author[1]{Valentine}{Herrmann}
\Author[1,2]{Rebecca}{Banbury Morgan}
\Author[3]{Ben}{Bond-Lamberty (confirm)}
\Author[4]{Susan}{Cook-Patton}
\Author[5]{Sandro}{Federici}
\Author[6]{Helene}{Muller-Landau}
\Author[7]{Camille}{Piponiot}
\Author[1]{Teagan}{Rogers}
\Author[5]{Valentyna}{Slivinska}
\Author[1,6 *]{Kristina J.}{Anderson-Teixeira}


\affil[1]{Center for Conservatiton Ecology, Smithsonian Conservation
Biology Institute, Front Royal, VA, United States}
\affil[2]{}
\affil[3]{}
\affil[4]{}
\affil[5]{IPCC Task Force on National Greenhouse Gas Inventories
Technical Support Unit, Institute for Global Environmental Strategies,
Hayama, Japan}
\affil[6]{Forest Global Earth Observatory, Smithsonian Tropical Research
Institute, Panama, Republic of Panama}
\affil[7]{CIRAD, Montpellier, France}

\runningtitle{IPCC forest C accounting with ForC}

\runningauthor{Williams et al.}


\correspondence{Kristina J.\ Anderson-Teixeira\ (teixeirak@si.edu)}



\received{}
\pubdiscuss{} %% only important for two-stage journals
\revised{}
\accepted{}
\published{}

%% These dates will be inserted by Copernicus Publications during the typesetting process.


\firstpage{1}

\maketitle


\begin{abstract}
The abstract goes here. It can also be on \emph{multiple lines}.
\end{abstract}


\copyrightstatement{The author's copyright for this publication is
transferred to institution/company.}


\textbf{Important}: Always double-check with the official manuscript
preparation guidelines at
\url{https://publications.copernicus.org/for_authors/manuscript_preparation.html},
especially the sections ``Technical instructions for LaTeX'' and
``Manuscript composition''. Please contact Daniel Nüst,
\texttt{daniel.nuest@uni-muenster.de}, with any problems.

\introduction[Introduction]

\textbf{(Forest are critical for climate change mitigation)}

\textbf{(Need for good data in international carbon accounting)}

\textbf{(Introduce EFDB \& ForC)}

Here, we: (1) clarify definitions of relevant carbon stocks and
increments (2) describe mapping of ForC to IPCC's EFDB, (3) describe
updates to ForC (ForC v4.0), (4) summarize the data in ForC that's
relevant to EFDB, identifying gaps, and (5) provide recommendations for
improving data collection, analysis, database, and accounting.

\section{Defining carbon stocks and incremenets}

For quantifying forest role in global C cycle, we ultimately care about:
(1) C stocks --stores of C that would be released to the atmosphere upon
and use change (2) C increments -- changes in those C stocks.

\subsection{Carbon stocks}

Forest ecosystem C stocks may be parsed into pools in various ways. IPCC
parses into biomass (aboveground and belowground), dead organic matter
(dead wood and litter), and soil organic matter (Table 1). Quantifying
these requires a one-time measurement.

\newpage

\textbf{Table 1: variables with definitions and measurement methods.}
Definitions from IPCC Table 1.1. (See Table 1.1 in IPCC guidance).
(\emph{Currently adding this as a figure (generated from original draft)
because kableExtra doesn't seem to work in this template, and I can't
quickly get the template format to work. Table that we want here is
``figures\_tables/C\_pools.csv''})

\begin{figure}
\includegraphics[width=12cm]{figures_tables/C_pools_fig} \caption{Table 1}\label{fig:unnamed-chunk-1}
\end{figure}

\begin{table*}[t]
\caption{This is a start at table 1 using the template format. }
\begin{tabular}{l c c c r}
\tophline

pool & subpool & definition & major sources of estimate variation & IPCC guidance \\
\middlehline
biomass & aboveground & all biomass of living vegetation, both woody and herbaceous, above the soil & allometry, min dbh & acceptable to exclude understory  \\

\bottomhline
\end{tabular}
\belowtable{I don't know how to adjust so that it doesn't run off the page.}
\end{table*}

\subsection{Carbon increments}

C increments are defined as the change over time, in annual increments,
in each C pool. These may be estimated as the difference between C
stocks at two time points, or as the difference between inputs and
outputs to the pool (i.e., fluxes). Quantifying these requires at least
two measurements.

Fluxes are the inputs and outputs to each pool.

\textbf{Figure: schematic illustrating fluxes in and out of each pool}

\section{Mapping ForC to EFDB}

\subsection{Carbon cycle variables}

\textbf{Table: variable mapping and equations-- give equations to
calculate IPCC variables from C cycle variables}

Define relationship among NEE, NEP, and delta.C., especially noting role
of harvest.

\subsection{Land use categories}

\section{Updates to ForC (ForC v4.0)}

To support export of data to EFDB, and to improve the overall quality of
the ForC database, we implemented some modest restructuring, resolved
duplicate records, and conducted quality control. This section describes
changes relative to ForC v2.0 \citep{anderson-teixeira_forc_2018}.

\subsection{ForC restructuring}

\begin{figure}[H]
\includegraphics[width=12cm]{figures_tables/ForC_updates_placeholder_fig} \caption{Table of changes to ForC fields (placeholder) }\label{fig:unnamed-chunk-2}
\end{figure}

\emph{(The above is a placeholder for the table located at
https://github.com/forc-db/ForC/blob/master/database\_management\_records/record\_of\_changes.csv,
which we'll need to format.)}

\subsection{Quality control measures}

Prior to releasing ForC v4.0, we executed several quality control
measures. First, to improve information on geographic coordinates, we
flagged and reviewed records with suspected low precision \emph{(Issue
\#29){[}https://github.com/forc-db/ForC/issues/229{]}}. Second, to
identify erroneous climate data\ldots{} \emph{(Issue
\#212){[}https://github.com/forc-db/ForC/issues/212{]}}.

\subsection{Resolving duplicates}

\section{Results}

\textbf{figure: map of relevant ForC data with underlying FAO ecozones}

\textbf{(summarize the data in ForC that's relevant to EFDB, identifying
gaps)}

dead wood and litter comparisons will be particularly interesting, as
IPCC values are based on just a handful of references for each climate
zone (table 2.2 in 2019 guidelines)

\section{Recommendations}

\subsection{Data collection and analysis needs}

\textbf{(Paragraph highlighting important gaps in variables / regions)}

Several variables of value to IPCC, including standing dead wood, woody
mortality, delta.agb, are not calculated and presented as frequently as
are AGB and ANPP\_woody, even though they can readily be derived from
the same census data. We recommend that researchers calculate and report
these, as specified below. Furthermore, there is an opportunity to fill
data gaps by calculating these from existing census data. For example,
the core census protocol of the Forest Global Earth Observatory
{[}ForestGEO; REFS{]} collects the data required to calculate standing
dead wood, woody mortality, and delta.agb, but these have not been
calculated and reported for all sites for which the appropriate number
of censuses are available (n=1 for standing dead wood, n=2 for woody
mortality and delta.agb) {[}but see REFS{]}.

A universal challenge in estimating biomass (living or dead) from forest
census data is applying appropriate allometries to convert DBH
measurements to biomass. \emph{(Camille/Helene can write this paragraph
easily.)}

\subsection{Data reporting needs}

We recommend that, unless they have some specific reason to do
otherwise, researchers calculate and report the values according to IPCC
standards:

\begin{itemize}
\item
  adopt common standards for variables like min diameter of deadwood,
  select soil sampling increments to include a cutoff at 30.
\item
  report 95\% CIs, SE, or STD and n
\item
  report C variables in article text, table, or SI table. EFDB cannot
  accept data digitized from figures
\end{itemize}

\textbf{Contributing data to ForC and/or EFDB directly will ensure its
broader impact.} The latter is more efficient for getting data to EFDB,
but does not get the data into ForC, where it can be more broadly
useful--for example, being used for basic science
\citep[e.g.,][]{banburymorgan_global_2021, anderson-teixeira_carbon_2021}
or model benchmarking \citep{fer_ecosystem_2021}.

\subsection{Database needs}

There are plenty of relevant, published data that are not included in
ForC. Systematic review of the literature could vastly improve data
coverage. \emph{(There are some efforts underway, including a few that
Susan can specify.)}

\subsection{IPCC}

An important challenge is that forests are changing rapidly, and data
collected a decaade ago may no longer be relevant, particularly in the
cases of C increments and fluxes.

Remote sensing biomass estimates include standing dead wood
\citep{duncanson_aboveground_2021}.

\conclusions[Conclusions]

The conclusion goes here. You can modify the section name with
\texttt{\textbackslash{}conclusions{[}modified\ heading\ if\ necessary{]}}.



\codedataavailability{use this to add a statement when having data sets
and software code
available} %% use this section when having data sets and software code available



%%%%%%%%%%%%%%%%%%%%%%%%%%%%%%%%%%%%%%%%%%
%% optional

%%%%%%%%%%%%%%%%%%%%%%%%%%%%%%%%%%%%%%%%%%

%%%%%%%%%%%%%%%%%%%%%%%%%%%%%%%%%%%%%%%%%%
\authorcontribution{(fill this in)} %% optional section

%%%%%%%%%%%%%%%%%%%%%%%%%%%%%%%%%%%%%%%%%%
\competinginterests{The authors declare no competing
interests.} %% this section is mandatory even if you declare that no competing interests are present

%%%%%%%%%%%%%%%%%%%%%%%%%%%%%%%%%%%%%%%%%%

%%%%%%%%%%%%%%%%%%%%%%%%%%%%%%%%%%%%%%%%%%
\begin{acknowledgements}
Thank you to all researchers who collected and published the data
contained in ForC, and to all research assistants and collaborators who
have helped to build the database. Funding for this study was provided
by The Nature Conservancy, the Institute for Global Environmental
Strategies, WLS(?)
\end{acknowledgements}

%% REFERENCES
%% DN: pre-configured to BibTeX for rticles

%% The reference list is compiled as follows:
%%
%% \begin{thebibliography}{}
%%
%% \bibitem[AUTHOR(YEAR)]{LABEL1}
%% REFERENCE 1
%%
%% \bibitem[AUTHOR(YEAR)]{LABEL2}
%% REFERENCE 2
%%
%% \end{thebibliography}

%% Since the Copernicus LaTeX package includes the BibTeX style file copernicus.bst,
%% authors experienced with BibTeX only have to include the following two lines:
%%
\bibliographystyle{copernicus}
\bibliography{references.bib}
%%
%% URLs and DOIs can be entered in your BibTeX file as:
%%
%% URL = {http://www.xyz.org/~jones/idx_g.htm}
%% DOI = {10.5194/xyz}


%% LITERATURE CITATIONS
%%
%% command                        & example result
%% \citet{jones90}|               & Jones et al. (1990)
%% \citep{jones90}|               & (Jones et al., 1990)
%% \citep{jones90,jones93}|       & (Jones et al., 1990, 1993)
%% \citep[p.~32]{jones90}|        & (Jones et al., 1990, p.~32)
%% \citep[e.g.,][]{jones90}|      & (e.g., Jones et al., 1990)
%% \citep[e.g.,][p.~32]{jones90}| & (e.g., Jones et al., 1990, p.~32)
%% \citeauthor{jones90}|          & Jones et al.
%% \citeyear{jones90}|            & 1990

\end{document}
