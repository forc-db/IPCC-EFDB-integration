%% Copernicus Publications Manuscript Preparation Template for LaTeX Submissions
%% ---------------------------------
%% This template should be used for copernicus.cls
%% The class file and some style files are bundled in the Copernicus Latex Package, which can be downloaded from the different journal webpages.
%% For further assistance please contact Copernicus Publications at: production@copernicus.org
%% https://publications.copernicus.org/for_authors/manuscript_preparation.html

%% copernicus_rticles_template (flag for rticles template detection - do not remove!)

%% Please use the following documentclass and journal abbreviations for discussion papers and final revised papers.

%% 2-column papers and discussion papers
\documentclass[, manuscript]{copernicus}



%% Journal abbreviations (please use the same for preprints and final revised papers)


% Advances in Geosciences (adgeo)
% Advances in Radio Science (ars)
% Advances in Science and Research (asr)
% Advances in Statistical Climatology, Meteorology and Oceanography (ascmo)
% Annales Geophysicae (angeo)
% Archives Animal Breeding (aab)
% ASTRA Proceedings (ap)
% Atmospheric Chemistry and Physics (acp)
% Atmospheric Measurement Techniques (amt)
% Biogeosciences (bg)
% Climate of the Past (cp)
% DEUQUA Special Publications (deuquasp)
% Drinking Water Engineering and Science (dwes)
% Earth Surface Dynamics (esurf)
% Earth System Dynamics (esd)
% Earth System Science Data (essd)
% E&G Quaternary Science Journal (egqsj)
% European Journal of Mineralogy (ejm)
% Fossil Record (fr)
% Geochronology (gchron)
% Geographica Helvetica (gh)
% Geoscience Communication (gc)
% Geoscientific Instrumentation, Methods and Data Systems (gi)
% Geoscientific Model Development (gmd)
% History of Geo- and Space Sciences (hgss)
% Hydrology and Earth System Sciences (hess)
% Journal of Bone and Joint Infection (jbji)
% Journal of Micropalaeontology (jm)
% Journal of Sensors and Sensor Systems (jsss)
% Magnetic Resonance (mr)
% Mechanical Sciences (ms)
% Natural Hazards and Earth System Sciences (nhess)
% Nonlinear Processes in Geophysics (npg)
% Ocean Science (os)
% Polarforschung - Journal of the German Society for Polar Research (polf)
% Primate Biology (pb)
% Proceedings of the International Association of Hydrological Sciences (piahs)
% Scientific Drilling (sd)
% SOIL (soil)
% Solid Earth (se)
% The Cryosphere (tc)
% Weather and Climate Dynamics (wcd)
% Web Ecology (we)
% Wind Energy Science (wes)

%% Please DO NOT add additional packages or LaTeX commands to the template. They
%% are not supported by Coperncius. LaTeX packages already
%% included in the copernicus.cls are:
%\usepackage[german, english]{babel}
%\usepackage{tabularx}
%\usepackage{cancel}
%\usepackage{multirow}
%\usepackage{supertabular}
%\usepackage{algorithmic}
%\usepackage{algorithm}
%\usepackage{amsthm}
%\usepackage{float}
%\usepackage{subfig}
%\usepackage{rotating}

% Pandoc citation processing

% The "Technical instructions for LaTex" by Copernicus require _not_ to insert any additional packages.
%%\usepackage{booktabs}
\usepackage{longtable}
\usepackage{array}
\usepackage{multirow}
\usepackage{wrapfig}
\usepackage{float}
\usepackage{colortbl}
\usepackage{pdflscape}
\usepackage{tabu}
\usepackage{threeparttable}
\usepackage{threeparttablex}
\usepackage[normalem]{ulem}
\usepackage{makecell}
\usepackage{xcolor}
%

\begin{document}

\title{Informing IPCC accounting of forest carbon using the global
forest carbon database (ForC v4.0)}


\Author[1]{Madison}{Williams}
\Author[1]{Valentine}{Herrmann}
\Author[1,2]{Rebecca}{Banbury Morgan}
\Author[3]{Ben}{Bond-Lamberty}
\Author[4]{Susan}{Cook-Patton}
\Author[5]{Helene}{Muller-Landau}
\Author[6]{Camille}{Piponiot}
\Author[1]{Teagan}{Rogers}
\Author[1,5 *]{Kristina J.}{Anderson-Teixeira}


\affil[1]{Center for Conservatiton Ecology, Smithsonian Conservation
Biology Institute, Front Royal, VA, United States}
\affil[2]{School of Geography, University of Leeds, Leeds, UK}
\affil[3]{Joint Global Change Research Institute, Pacific Northwest
National Laboratory, College Park, MD, United States}
\affil[4]{The Nature Conservancy; Arlington VA 22203, USA}
\affil[5]{Forest Global Earth Observatory, Smithsonian Tropical Research
Institute, Panama, Republic of Panama}
\affil[6]{CIRAD, Montpellier, France}

\runningtitle{IPCC forest C accounting with ForC}

\runningauthor{Williams et al.}


\correspondence{Kristina J.\ Anderson-Teixeira\ (teixeirak@si.edu)}



\received{}
\pubdiscuss{} %% only important for two-stage journals
\revised{}
\accepted{}
\published{}

%% These dates will be inserted by Copernicus Publications during the typesetting process.


\firstpage{1}

\maketitle


\begin{abstract}
Forests are critical for climate change mitigation and consitute a
substantial portion of planned emissions reductions under the 2015 Paris
Agreement. Yet, the efficacy of greenhouse gas mitigation planning and
reporting is dependent upon the quality of available emission factors
data, including forest carbon (C) stocks and changes therein. Tens of
thousands of relevant forest C estimates have been published, yet are
not readily accesible to the practitioners compiling national greenhouse
gas inventories. Many of these data have, however, been compiled in the
Global Forest C database (ForC; https://forc-db.github.io/) and stand to
be of value to greenhouse gas accounting if made available through the
Emission Factor Database (EFDB) of the International Panel on Climate
Change (IPCC). Here, we develop and document a process for
semi-automated transfer of data from ForC into the EFDB, assess the data
available and transferred to date, and provide recommendations for
improving forest data collection, analysis, and reporting to improve
accounting of forest-sector greenhoouse gas emissions and removals. We
begin by reconciling terminology and mapping ForC fields into EFDB. This
process required some updates to the ForC database structure, leading to
the release of a new version of ForC (v4.0; described here). At the time
of writing, ForC contained \#\# values that would qualify for inclusion
in the EFDB, \#\# of which have been transferred to date. (Some analysis
of representation/ gaps.) In the future, forest C estimates in EFDB can
be improved through targetted research to fill critical gaps, reporting
of information required by IPCC, and continued submission of data from
scientific publications to the EFDB.
\end{abstract}


\copyrightstatement{The author's copyright for this publication is
transferred to institution/company.}


\introduction[Introduction]

Forests are critical to management of atmospheric concentrations of the
greenhouse gas carbon dioxide (CO\textsubscript{2}), and thereby climate
change. In recent decades, CO\textsubscript{2} uptake by forests,
woodlands, and savannas has exceeded releases from deforestation and
other severe disturbances, resulting in a net carbon CO\textsubscript{2}
sink of \textasciitilde0.88 Gt C yr\textsuperscript{-1} \citep[all
biomes with trees,][]{xu_changes_2021} to \textasciitilde1.6 Gt C
yr\textsuperscript{-1} \citep[forests only,][]{harris_global_2021}. This
has offset an estimated 10\% to 18\% of anthropogenic
CO\textsubscript{2} emissions from fossil fuels and cement
\citep{xu_changes_2021, harris_global_2021}, dramatically slowing the
pace of atmospheric CO\textsubscript{2} accumulation and climate change.
Going into the future, the fate of this important CO\textsubscript{2}
sink is highly uncertain, depending both upon forest responses to
climate change, which are likely to reduce the sink strength
\citep{refs}, and on human conservation, restoration, and management of
forests \citep{refs}.

Reflecting their strong influence on Earth's climate, forests play a
central role in international plans for climate change mitigation under
the Paris Agreement \citep{unfccc_adoption_2015}. Forest conservation,
reforestation, and improved sustainable management all have significant
-- and relatively cost-effective -- potential as climate change
mitigation options, with conservation and reforestation having the
fourth and fifth largest net emission reduction potentials or all
mitigation options \citep{ipcc_summary_2022}. As of 2016, forest-based
mitigation accounted for 26\% of total planned greenhouse gas mitigation
within Nationally Determined Contributions under the Paris Agreement
\citep{grassi_key_2017}. Yet, envisioned forest-based climate change
mitigation initiatives do not always correspond to actual emission
reductions through on-the-ground implementation \citep{refs}. One
critical need for ensuring that forest-based climate change mitigation
initiatives are effective is realistic planning, underlain by solid
scientific data \citep{anderson-teixeira_effective_2022}.

The International Panel on Climate Change (IPCC) provides guidance for
national greenhouse gas inventories for reporting to the United Nations
Framework Convention on Climate Change
\citep[UNFCCC,][]{ipcc_2019_2019}. Under this guidance, greenhouse gas
inventories include all managed land, including most of the world's
forest land \citep{ogle_delineating_2018}. The IPCC inventory guidelines
include specific instructions for accounting for greenhouse gas (mainly
CO\textsubscript{2}) exchanges between forest land and the atmosphere
\citep{ipcc_agriculture_2006, ipcc_2019_2019}. This guidance has
improved over the years as more of the relevant underlying data has
become available
\citep{requenasuarez_estimating_2019, rozendaal_aboveground_2022}, but
there remains room for continuous improvement as the science advances.
For example, the year following the release of the latest IPCC
guidelines, \citet{cook-patton_mapping_2020} found that the latest
default rates may underestimate rates of C accumulation in regrowth
forests by 32\% on average and fail to capture eight-fold variation
within ecozones. In addition, \citet{cuni-sanchez_high_2021} found that
aboveground C stocks in mature African tropical montane forests were
two-thirds higher than the IPCC default values for these forests. This
rapid evolution of scientific information on the climate mitigation
potential of forests is beneficial to climate mitigation efforts, but
requires improved mechanisms for communicating the latest information
from scientific researchers to the practitioners who need reliable
estimates for greenhouse gas mitigation planning. Moreover, high
variability of forest C cycling within ecozones
\citep[e.g.,][]{cook-patton_mapping_2020, refs} implies that it is
useful for those compiling national greenhouse gas inventories to have
access to locally-specific information, when available. To improve the
data accessible for C accounting, the IPCC created the Emission Factor
Database (EFDB; https://www.ipcc-nggip.iges.or.jp/EFDB/main.php), which
is intended as a recognized library of emission factors and other
parameters that can be used for estimating greenhouse gas emissions and
removals.

\textbf{To ensure that planned emissions reductions are realistic,
high-quality estimates of forest C stocks and fluxes must be publicly
accessible.} The EFDB is intended as a recognized library, and can be
used both for efforts to tally a nation's intended or accomplished
greenhouse gas reductions, or as a basis of comparison for external
parties to evaluate these inventories.

\textbf{The Global Forest Carbon Database, ForC, is the largest
collection of published estimates of forest carbon stocks, increments,
and annual fluxes
\citep{anderson-teixeira_forc_2018, anderson-teixeira_carbon_2021}.}
\emph{(add stats/ details, maybe record of how ForC has grown over
time)} As such, ForC is positioned to improve forest C accounting
through the transfer of data to EFDB. The purpose of this publication is
to document that process and provide recommendations for future
improvements.

\textbf{Here, we} (1) review IPCC definitions of relevant carbon stocks
and increments (2) describe mapping of ForC to IPCC's EFDB, (3) describe
updates to ForC (ForC v4.0), (4) summarize the data in ForC that's
relevant to EFDB, identifying gaps, and (5) provide recommendations for
improving data collection, analysis, database, and accounting.

\section{IPCC definitions of carbon stocks and incremenets}

For quantifying forest role in global C cycle, we ultimately care about:
(1) C stocks -- stores of C that would be vulnerable to release to the
atmosphere upon land use change (2) C increments -- changes in those C
stocks.

\subsection{Carbon stocks}

Forest ecosystem C stocks may be parsed into pools in various ways. IPCC
parses into biomass (aboveground and belowground), dead organic matter
(dead wood and litter), and soil organic matter
(Table~@ref(table\_variables)). Quantifying these requires a one-time
measurement.

\begin{table}

\caption{\label{tab:table_variables}\textbf{Variables with definitions and measurement methods.} Definitions from IPCC Table 1.1. (See Table 1.1 in IPCC guidance).}
\centering
\begin{tabu} to \linewidth {>{\raggedright}X>{\raggedright}X>{\raggedright}X>{\raggedright}X}
\hline
\textbf{pool} & \textbf{definition} & \textbf{major sources of estimate variation} & \textbf{IPCC guidance}\\
\hline
aboveground biomass & all biomass of living vegetation, both woody and herbaceous, above the soil & allometry, min dbh & acceptable to exclude understory\\
\hline
belowground biomass & all biomass of live roots & allometry, min dbh, assumed ratio of belowground to aboveground biomass (IPCC table 4.4) & fine roots may be excluded when they cannot be distinguished empirically from soil organic matter or litter\\
\hline
dead wood & all non-living woody biomass not contained in the litter, either standing, lying on the ground, or in the soil & min dbh,  ... & default min dbh = 10cm, but may be chosen by country\\
\hline
litter & all non-living biomass with a size greater than the limit for soil organic matter  and less than the minimum diameter chosen for dead wood, lying dead, in various states of decomposition above or within the mineral or organic soil & min dbh for dead wood , .. & includes entire O horizon\\
\hline
soil organic matter & organic carbon in mineral soils to a specified depth & sampling depth & default sampling depth = 30cm, but may be chosen by country\\
\hline
\end{tabu}
\end{table}

\subsubsection{Biomass}

Biomass includes living vegetation, above- and below-ground.

The IPCC defines aboveground biomass as ``all biomass of living
vegetation, both woody and herbaceous, above the soil including stems,
stumps, branches, bark, seeds, and foliage'' {[}{]}.

Belowground biomass is defined as ``all biomass of live roots'' {[}{]}.

\subsubsection{Dead Organic Matter}

Dead organic matter includes all non-living biomass that is not within
the mineral soil layer and smaller than the litter size threshold.

Dead wood is defined as\ldots{}

Litter is defined as including " all non-living biomass with a diameter
less than a minimum diameter chosen by the country (for example 10 cm),
lying dead, in various states of decomposition above the mineral or
organic soil. This includes litter (OL), fumic (OF), and humic (OH)
layers. Live fine roots (of less than the suggested diameter limit for
belowground biomass) are included in litter where they cannot be
distinguished from it empirically." (2003 IPCc GPG for LULUCF
(https://www.ipcc-nggip.iges.or.jp/public/gpglulucf/gpglulucf\_files/Glossary\_Acronyms\_BasicInfo/Glossary.pdf)

\subsubsection{Soil Organic Matter}

Soil organic matter is defined as ``Includes organic carbon in mineral
and organic soils (including peat) to a specified depth chosen by the
country and applied consistently through the time series. Live fine
roots (of less than the suggested diameter limit for belowground
biomass) are included with soil organic matter where they cannot be
distinguished from it empirically.''(2003 IPCc GPG for LULUCF
(https://www.ipcc-nggip.iges.or.jp/public/gpglulucf/gpglulucf\_files/Glossary\_Acronyms\_BasicInfo/Glossary.pdf)

\subsection{Carbon increments}

C increments are defined as the change over time, in annual increments,
in each C pool. These may be estimated as the difference between C
stocks at two time points, or as the difference between inputs and
outputs to the pool (i.e., fluxes). Quantifying these requires at least
two measurements. (\emph{But, Can carbon increments be inferred from a
single measure and a known age (i.e., the approach we used in GROA)?})
Fluxes are the inputs and outputs to each pool.

\clearpage
\begin{figure}
\includegraphics[width=15cm]{figures_tables/C_variable_mapping} \caption{\textbf{Schematic illustrating the carbon pools quantified under IPCC accounting; ForC variables corresponding to the stock, increment, influx and outflux; and relationships among them.} In many cases, the match of ForC variables to IPCC criteria depends upon measurement protocols (e.g., minimum DBH). Additional caveats are as follows: 1- assumes that change in foliage biomass is negligible (see note 7); 2- incomplete: excludes large branch fall; also, under IPCC definitions, outflux from aboveground biomass should include all sizes, influx to deadwood should include only above the minimum diameter chosen for dead wood; 3- incomplete: excludes belowground components;  4-incomplete: excludes breakage into pieces less than dead wood threshold size; 5-incomplete: excludes woody mortality of stems <10 cm DBH, decomposition of dead wood (aboveground and coarse roots) into sizes classified as litter, may exclude branch fall; 6- measurements often limited to decomposition of relatively fine litter and may exclude branches and stems below the dead wood size threshold and/or the more decomposed layers of the O horizon; 7 - foliage production is generally measured by collecting leaf-fall, a method that assumes that the influx = outflux (foliage biomass is roughly constant year-to-year); 8 - excludes branch fall, which is necessary for a full accounting of woody productivity but is typically assumed negligible for calculations of net biomass change.}\label{fig:unnamed-chunk-1}
\end{figure}

\section{Mapping ForC to EFDB}

ForC data is incredibly valuable to EFDB and there is data which is
included in the ForC database that does not meet EFDB standards. There
were two main EFDB guidelines which limits the amount of data we could
transfer. EFDB will not accept data which has been digitized(from graph)
and ForC does.

\subsection{Carbon cycle variables}

Mapping of variables is shown in Fig. 1

\subsection{Land use categories}

Documented at
https://github.com/forc-db/IPCC-EFDB-integration/blob/main/doc/ForC-EFDB\_mapping/defining\_land\_subcategory.md,
https://github.com/forc-db/IPCC-EFDB-integration/blob/main/doc/ForC-EFDB\_mapping/IPCC\_LandUse\_mapping.csv,
and in
\href{https://github.com/forc-db/IPCC_database_integration/issues/8}{issue
\#8}.

The UNFCCC requires greenhouse gas reporting for all managed lands in a
country, where management is defined as ``human interventions and
practices have been applied to perform production, ecological or social
functions'' {[}\emph{IPCC 2006 full report REF}{]}. This definition is
applied differently across countries, and is not clearly defined by the
majority of governments \citep{ogle_delineating_2018}. Given this, and
because the IPCC definition of management does not necessarily match
that which would be reported in scientific publications and hence in
ForC, we do not transfer any classification of land management status
from ForC to the EFDB, but do provide auxiliary info that may be useful
in making this determination (e.g., geographical location).

\section{Updates to ForC (ForC v4.0)}

To support export of data to EFDB, and to improve the overall quality of
the ForC database, we defined \#\# new variables, implemented some
modest restructuring, resolved duplicate records, and conducted quality
control. This section describes changes relative to ForC v2.0
\citep{anderson-teixeira_forc_2018}.

\subsection{Defining new variables}

We added eleven increment variables to the set of named and defined
variables (or 22, counting \_OM and \_C versions), which previously
included only one (aboveground biomass increment, \emph{delta.agb}).
\emph{(https://github.com/forc-db/IPCC-EFDB-integration/issues/6)} These
are directly related to C stocks as previously defined in ForC, with
``\emph{delta.}'' added in front of the variable name.

Although these variables currently lack records, the structure exists
such that records can be populated over time.

To provide better definition of the previously existing variable
\emph{organic.layer}, which has a nebulous definition that reflects the
varied definitions adopted by original studies, we added two clearly
defined variables: \emph{litter} (relatively undecomposed plant
material/ OL horizon), and \emph{O.horizon} (entire O-horizon, including
\emph{litter} (OL)).

\subsection{ForC restructuring}

\subsection{Quality control measures}

Prior to releasing ForC v4.0, we executed several quality control
measures. First, we implemented a system of continuous integration using
GitHub Actions (\emph{sensu} Kim et al.~in prep) to run some automatic
checks any time the master data files are updated. Second, to improve
information on geographic coordinates, we flagged and reviewed records
with suspected low precision \emph{(Issue
\#29){[}https://github.com/forc-db/ForC/issues/229{]}}. Third, to
identify erroneous climate data\ldots{} \emph{(Issue
\#212){[}https://github.com/forc-db/ForC/issues/212{]}}.

\subsection{Resolving duplicates}

\section{Results}

\textbf{figure: map of relevant ForC data with underlying FAO ecozones}

\textbf{(summarize the data in ForC that's relevant to EFDB, identifying
gaps)}

dead wood and litter comparisons will be particularly interesting, as
IPCC values are based on just a handful of references for each climate
zone (table 2.2 in 2019 guidelines)

\section{Recommendations}

(\emph{strongly flag both useful variables that the EFDB does not track
and useful variables that papers fail to include that EFDB needs})

\subsection{Data collection and analysis needs}

\textbf{(Paragraph highlighting important gaps in variables / regions)}

Several variables of value to IPCC, including standing dead wood, woody
mortality, delta.agb, are not calculated and presented as frequently as
are AGB and ANPP\_woody, even though they can readily be derived from
the same census data. We recommend that researchers calculate and report
these, as specified below. Furthermore, there is an opportunity to fill
data gaps by calculating these from existing census data. For example,
the core census protocol of the Forest Global Earth Observatory
{[}ForestGEO; REFS{]} collects the data required to calculate standing
dead wood, woody mortality, and delta.agb, but these have not been
calculated and reported for all sites for which the appropriate number
of censuses are available (n=1 for standing dead wood, n=2 for woody
mortality and delta.agb) {[}but see REFS{]}.

A universal challenge in estimating biomass (living or dead) from forest
census data is applying appropriate allometries to convert DBH
measurements to biomass. \emph{(Camille/Helene can write this paragraph
easily.)}

\subsection{Data reporting needs}

We recommend that, unless they have some specific reason to do
otherwise, researchers calculate and report the values according to IPCC
standards:

\begin{itemize}
\item
  adopt common standards for variables like min diameter of deadwood,
  select soil sampling increments to include a cutoff at 30.
\item
  report 95\% CIs, SE, or STD and n
\item
  report C variables in article text, table, or SI table. EFDB cannot
  accept data digitized from figures
\item
  present calculations of all variables that would be useful to IPCC.
  EFDB requires that data in the database be presented in the original
  article, and as such cannot accept subsequent calculations. For
  example, if aboveground biomass and total biomass are presented, but
  root biomass is not presented, root biomass cannot be subsequently
  calculated and sent to EFDB. Similarly, fine and coarse root biomass
  can't be summed; soil carbon can't be summed across depth increments,
  etc.
\end{itemize}

For data synthesis projects, compilation can only be useful to the EFDB
if they include all the required, along with transparent description on
the methodology applied to derive emission factors (or have a brief
description and a reference to the original source) and the original
emission factor values are present (not modified/rounded).

\textbf{Contributing data to ForC and/or EFDB directly will ensure its
broader impact.} The latter is more efficient for getting data to EFDB,
but does not get the data into ForC, where it can be more broadly
useful--for example, being used for basic science
\citep[e.g.,][]{banburymorgan_global_2021, anderson-teixeira_carbon_2021}
or model benchmarking \citep{fer_ecosystem_2021}.

\subsection{Database needs}

There are plenty of relevant, published data that are not included in
ForC. Systematic review of the literature could vastly improve data
coverage. \emph{(There are some efforts underway, including a few that
Susan can specify.)}

\subsection{IPCC}

An important challenge is that forests are changing rapidly, and data
collected a decaade ago may no longer be relevant, particularly in the
cases of C increments and fluxes.

Remote sensing biomass estimates include standing dead wood
\citep{duncanson_aboveground_2021}.

\section{Conclusions}

\clearpage

\section{Appendix A}

\begin{longtabu} to \linewidth {>{\raggedright}X>{\raggedright}X>{\raggedright}X>{\raggedright}X>{\raggedright}X}
\caption{\label{tab:table_ForCfieldmapping}\textbf{Mapping of ForC fields to EFDB.} See footnotes at end of table (still need to be properly inserted). }\\
\hline
\textbf{ForC table} & \textbf{ForC field} & \textbf{EFDB field} & \textbf{Usage} & \textbf{Required}\\
\hline
\endfirsthead
\caption[]{\textbf{Mapping of ForC fields to EFDB.} See footnotes at end of table (still need to be properly inserted).  \textit{(continued)}}\\
\hline
\textbf{ForC table} & \textbf{ForC field} & \textbf{EFDB field} & \textbf{Usage} & \textbf{Required}\\
\hline
\endhead
Measurements & measurement.ID & Other Properties & direct mapping & (no)\\
\hline
 & dominant.life.form & 1996 Source/Sink Categories, 2006 Source/Sink Categories & used to determine land subcategories (see defining\_land\_subcategory.md) & yes\\
\hline
 & stand.age & 1996 Source/Sink Categories, 2006 Source/Sink Categories, Parameters/ Conditions & used to determine land subcategories (see defining\_land\_subcategory.md), directly listed in Parameters/ Conditions & (yes)\\
\hline
 & dominant.veg, veg.notes, min.dbh & Parameters/ Conditions & direct mapping/ linking to dominant.veg description & no\\
\hline
 & variable.name & - & link to variable info in ForC\_variables table & yes\\
\hline
 & date / start.date, end.date & Other Properties & direct mapping & no\\
\hline
 & mean & Value & direct mapping & yes\\
\hline
 & mean.in.original.units & Value in Common Units & direct mapping & yes\\
\hline
 & original.units & Common Unit & direct mapping & yes\\
\hline
 & lower95\%CI, upper 95\%CI, se, sd and n & Lower Confidence Limit, Upper Confidence Limit & direct or calculated & (yes)\\
\hline
 & depth, covariate\_1, cov\_1.value, covariate\_2, cov\_2.value & Other Properties & direct mapping & no\\
\hline
 & allometry\_1, allometry\_2 & Comments from Data Provider & link to biomass allometry source, when provided & no\\
\hline
 & data.location.within.source & - & confirm that data weren't digitized, facilitate finding data in original publication & yes\\
\hline
 & ForC.investigator & Data Provider, Data Provider Contact & link to Data Provider, Data Provider Contact info & yes\\
\hline
Sites & site.ID, sites.sitename & Other Properties & direct mapping & (no)\\
\hline
 & lat, lon & Region/Regional conditions & direct mapping; used to extract continent, Koeppen, and FAO.ecozone & (no)\\
\hline
 & country, state, city, masl,  mat, map & Region/Regional conditions & direct mapping & no\\
\hline
 & continent, Koeppen & Region/Regional conditions & direct mapping & auto\\
\hline
 & soil.texture, sand, silt, clay, soil.classification & Parameters/ Conditions & direct mapping & no\\
\hline
 & FAO.ecozone & Parameters/ Conditions & direct mapping & auto\\
\hline
History & date, hist.cat, hist.type & 1996 Source/Sink Categories, 2006 Source/Sink Categories, Abatement/Control technologies & used to determine distmrs.type for Source/Sink Categories, generate list of events for Abatement/Control technologies & (yes)/no**\\
\hline
 & plot.area & Other Properties & direct mapping & no\\
\hline
Plots & plot.ID, plot.name & Other Properties & direct mapping & (no)\\
\hline
 & distmrs.type & 1996 Source/Sink Categories, 2006 Source/Sink Categories & used to determine land subcategories (see defining\_land\_subcategory.md) & auto\\
\hline
 & distmrs.type, distmrs.year, regrowth.type, regrowth.year & Other Properties & direct mapping & auto\\
\hline
PFT & description & Parameters/ Conditions & direct mapping & auto\\
\hline
variables & variable.type & Gases & For stocks in unit of organic matter, gases include CO2, CO, CH4, NO, NO2, N2O. For increments, fluxes, and stocks in units of C, gases includes only CO2. & auto\\
\hline
 & variable.name & C pool, Equation & link to C pool, Equation & auto\\
\hline
 & description & Description & direct mapping & auto\\
\hline
 & extended.description & Other Properties & direct mapping & auto\\
\hline
 & units & Unit (ID) & link to IPCC units & auto\\
\hline
Citations & citation.citation & Full Technical Reference & direct mapping & yes/auto\\
\hline
 & citation.language & Reference Language & direct mapping & yes/auto\\
\hline
 & citation.url & URL & direct mapping & no/auto\\
\hline
 & citation.abstract & Abstract in English & direct mapping & no/auto\\
\hline
 & source.type & Source of Data & direct mapping & yes\\
\hline
\end{longtabu}

`Required' field indicates whether the field is required by EFDB: yes =
value required; (yes) = input required, missing value acceptable if not
reported; auto = present within ForC infrasructure, and therefore will
always be exported to EFDB ; (no) = not required for EFDB, but required
for ForC and therefore will always be exported to EFDB; no = not
required, but exported to EFDB when a value is present.

** `(yes)' for most recent severe disturbance; `no' for other history
events

\clearpage

\section{Appendix B}

\begin{longtabu} to \linewidth {>{\raggedright}X>{\raggedright}X>{\raggedright}X>{\raggedright}X>{\raggedright}X}
\caption{\label{tab:table_ForCchanges}\textbf{Table of changes to ForC fields.}}\\
\hline
\textbf{Table} & \textbf{Column} & \textbf{Description} & \textbf{Changes} & \textbf{Motivation}\\
\hline
\endfirsthead
\caption[]{\textbf{Table of changes to ForC fields.} \textit{(continued)}}\\
\hline
\textbf{Table} & \textbf{Column} & \textbf{Description} & \textbf{Changes} & \textbf{Motivation}\\
\hline
\endhead
Sites & coordinates.precision & Precision of geographic coordinates, as reported by source or estimated from maps. & field added & allow identification of records with poor coordinate precision\\
\hline
Measurements & data.location.within.source & Location of data within the source listed in citation.ID. & field added & facilitate review, ensure traceability\\
\hline
 & sd, se, lower95\%CI, upper 95\%CI & Standard deviation, standard error, and lower and upper 95 percent confidence intvervals, respectively. & replaces `stat` and `stat.name` & cleaner format; ability to handle assymetrical 95 percent confidence intervals\\
\hline
 & mean.in.original.units, original.units & mean value and units presented in original publication & fields added & provide IPCC with original units, reduce errors/improve reproducibility\\
\hline
 & C.conversion.factor & Assumed/ measured C content of organic matter used to convert organic matter to C. & field added & track units conversion, allow back-calculation of OM if conversion factor deemed inappropriate\\
\hline
PFT & description & Definition of the pftcode at the community level. Differs from individual level in that properly describes mixed plant functional types. & field added & \\
\hline
 & description.individual & Definition of the pftcode at the individual plant level. & field name change (previously `description`) & \\
\hline
Citations & citation.citation & Full citation. Most of these records are automatically generated in R based upon DOI lookup. & field added & field required by IPCC\\
\hline
 & citation.language & Language of original publication, automatically generated based on the title and abstract, with some manual entries and corrections. & field added & field required by IPCC\\
\hline
 & citation.url & URL of original publication, generally retrieved automatically via URL lookup. & field added & field required by IPCC\\
\hline
 & citation.abstract & Abstract, generally retrieved automatically via DOI lookup. & field added & field required by IPCC\\
\hline
 & source.type & citation source type & field added & field required by IPCC\\
\hline
 & pdf.in.repository & Indicates whether pdf of original study has been retrieved and saved in ForC's reference repository & field added & housekeeping\\
\hline
 & EFDB.ready & Indicates whether data have been checked for export to EFDB. & field added & housekeeping\\
\hline
\end{longtabu}



\codedataavailability{use this to add a statement when having data sets
and software code
available} %% use this section when having data sets and software code available



%%%%%%%%%%%%%%%%%%%%%%%%%%%%%%%%%%%%%%%%%%
%% optional

%%%%%%%%%%%%%%%%%%%%%%%%%%%%%%%%%%%%%%%%%%
\appendix
\section{Mapping ForC to EFDB}

CURRENT TABLE 2 GOES HERE

\section{Updates to ForC}

CURRENT TABLE 3 GOES HERE
\noappendix

%%%%%%%%%%%%%%%%%%%%%%%%%%%%%%%%%%%%%%%%%%
\authorcontribution{(fill this in)} %% optional section

%%%%%%%%%%%%%%%%%%%%%%%%%%%%%%%%%%%%%%%%%%
\competinginterests{The authors declare no competing
interests.} %% this section is mandatory even if you declare that no competing interests are present

%%%%%%%%%%%%%%%%%%%%%%%%%%%%%%%%%%%%%%%%%%

%%%%%%%%%%%%%%%%%%%%%%%%%%%%%%%%%%%%%%%%%%
\begin{acknowledgements}
Thank you to all researchers who collected and published the data
contained in ForC, and to all research assistants and collaborators who
have helped to build the database. Funding for this study was provided
by Bezos Earth Fund to The Nature Conservancy, the Institute for Global
Environmental Strategies, WLS(?)
\end{acknowledgements}

%% REFERENCES
%% DN: pre-configured to BibTeX for rticles

%% The reference list is compiled as follows:
%%
%% \begin{thebibliography}{}
%%
%% \bibitem[AUTHOR(YEAR)]{LABEL1}
%% REFERENCE 1
%%
%% \bibitem[AUTHOR(YEAR)]{LABEL2}
%% REFERENCE 2
%%
%% \end{thebibliography}

%% Since the Copernicus LaTeX package includes the BibTeX style file copernicus.bst,
%% authors experienced with BibTeX only have to include the following two lines:
%%
\bibliographystyle{copernicus}
\bibliography{references.bib}
%%
%% URLs and DOIs can be entered in your BibTeX file as:
%%
%% URL = {http://www.xyz.org/~jones/idx_g.htm}
%% DOI = {10.5194/xyz}


%% LITERATURE CITATIONS
%%
%% command                        & example result
%% \citet{jones90}|               & Jones et al. (1990)
%% \citep{jones90}|               & (Jones et al., 1990)
%% \citep{jones90,jones93}|       & (Jones et al., 1990, 1993)
%% \citep[p.~32]{jones90}|        & (Jones et al., 1990, p.~32)
%% \citep[e.g.,][]{jones90}|      & (e.g., Jones et al., 1990)
%% \citep[e.g.,][p.~32]{jones90}| & (e.g., Jones et al., 1990, p.~32)
%% \citeauthor{jones90}|          & Jones et al.
%% \citeyear{jones90}|            & 1990

\end{document}
