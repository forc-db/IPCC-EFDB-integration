%% Copernicus Publications Manuscript Preparation Template for LaTeX Submissions
%% ---------------------------------
%% This template should be used for copernicus.cls
%% The class file and some style files are bundled in the Copernicus Latex Package, which can be downloaded from the different journal webpages.
%% For further assistance please contact Copernicus Publications at: production@copernicus.org
%% https://publications.copernicus.org/for_authors/manuscript_preparation.html

%% copernicus_rticles_template (flag for rticles template detection - do not remove!)

%% Please use the following documentclass and journal abbreviations for discussion papers and final revised papers.

%% 2-column papers and discussion papers
\documentclass[, manuscript]{copernicus}



%% Journal abbreviations (please use the same for preprints and final revised papers)

% Advances in Geosciences (adgeo)
% Advances in Radio Science (ars)
% Advances in Science and Research (asr)
% Advances in Statistical Climatology, Meteorology and Oceanography (ascmo)
% Annales Geophysicae (angeo)
% Archives Animal Breeding (aab)
% Atmospheric Chemistry and Physics (acp)
% Atmospheric Measurement Techniques (amt)
% Biogeosciences (bg)
% Climate of the Past (cp)
% DEUQUA Special Publications (deuquasp)
% Drinking Water Engineering and Science (dwes)
% Earth Surface Dynamics (esurf)
% Earth System Dynamics (esd)
% Earth System Science Data (essd)
% E&G Quaternary Science Journal (egqsj)
% EGUsphere (egusphere) | This is only for EGUsphere preprints submitted without relation to an EGU journal.
% European Journal of Mineralogy (ejm)
% Fossil Record (fr)
% Geochronology (gchron)
% Geographica Helvetica (gh)
% Geoscience Communication (gc)
% Geoscientific Instrumentation, Methods and Data Systems (gi)
% Geoscientific Model Development (gmd)
% History of Geo- and Space Sciences (hgss)
% Hydrology and Earth System Sciences (hess)
% Journal of Bone and Joint Infection (jbji)
% Journal of Micropalaeontology (jm)
% Journal of Sensors and Sensor Systems (jsss)
% Magnetic Resonance (mr)
% Mechanical Sciences (ms)
% Natural Hazards and Earth System Sciences (nhess)
% Nonlinear Processes in Geophysics (npg)
% Ocean Science (os)
% Polarforschung - Journal of the German Society for Polar Research (polf)
% Primate Biology (pb)
% Proceedings of the International Association of Hydrological Sciences (piahs)
% Safety of Nuclear Waste Disposal (sand)
% Scientific Drilling (sd)
% SOIL (soil)
% Solid Earth (se)
% The Cryosphere (tc)
% Weather and Climate Dynamics (wcd)
% Web Ecology (we)
% Wind Energy Science (wes)

% Pandoc citation processing

% The "Technical instructions for LaTex" by Copernicus require _not_ to insert any additional packages.
% 
% tightlist command for lists without linebreak
\providecommand{\tightlist}{%
  \setlength{\itemsep}{0pt}\setlength{\parskip}{0pt}}


%%\usepackage{booktabs}
\usepackage{longtable}
\usepackage{array}
\usepackage{multirow}
\usepackage{wrapfig}
\usepackage{float}
\usepackage{colortbl}
\usepackage{pdflscape}
\usepackage{tabu}
\usepackage{threeparttable}
\usepackage{threeparttablex}
\usepackage[normalem]{ulem}
\usepackage{makecell}
\usepackage{xcolor}
%
\begin{document}


\title{Informing IPCC accounting of forest carbon using the global
forest carbon database (ForC v4.0)}


\Author[1,2 *]{Kristina J.}{Anderson-Teixeira}
\Author[1]{Valentine}{Herrmann}
\Author[1]{Madison}{Williams}
\Author[1]{Teagan}{Rogers}
\Author[1,3]{Rebecca}{Banbury Morgan}
\Author[4]{Ben}{Bond-Lamberty}
\Author[5]{Susan}{Cook-Patton}
\Author[2]{Helene}{Muller-Landau}
\Author[6]{Camille}{Piponiot}


\affil[1]{Center for Conservation Ecology, Smithsonian's National Zoo \&
Conservation Biology Institute, Front Royal, VA, United States}
\affil[2]{Forest Global Earth Observatory, Smithsonian Tropical Research
Institute, Panama, Republic of Panama}
\affil[3]{School of Geography, University of Leeds, Leeds, UK}
\affil[4]{Joint Global Change Research Institute, Pacific Northwest
National Laboratory, College Park, MD, United States}
\affil[5]{The Nature Conservancy; Arlington VA 22203, USA}
\affil[6]{CIRAD, Montpellier, France}

\runningtitle{IPCC forest C accounting with ForC}

\runningauthor{Anderson-Teixeira et al.}


\correspondence{Kristina J.\ Anderson-Teixeira\ (teixeirak@si.edu)}



\received{}
\pubdiscuss{} %% only important for two-stage journals
\revised{}
\accepted{}
\published{}

%% These dates will be inserted by Copernicus Publications during the typesetting process.


\firstpage{1}

\maketitle






\begin{center}\rule{0.5\linewidth}{0.5pt}\end{center}

\emph{THIS IS AN IN-PREP MANUSCRIPT.}

\begin{center}\rule{0.5\linewidth}{0.5pt}\end{center}

\textbf{Abstract.} Forests are critical for climate change mitigation
and constitute a substantial portion of planned emissions reductions
under the 2015 Paris Agreement. Yet, the efficacy of greenhouse gas
mitigation planning and reporting is dependent upon the quality of
available emission factors data, including forest carbon (C) stocks and
changes therein. Tens of thousands of relevant forest C estimates have
been published, yet are not readily accesible to the practitioners
compiling national greenhouse gas inventories. Many of these data have,
however, been compiled in the Global Forest C database (ForC;
https://forc-db.github.io/) and stand to be of value to greenhouse gas
accounting if made available through the Emission Factor Database (EFDB)
of the International Panel on Climate Change (IPCC). Here, we develop
and document a process for semi-automated transfer of data from ForC
into the EFDB, assess the data available and transferred to date, and
provide recommendations for improving forest data collection, analysis,
and reporting to improve accounting of forest-sector greenhoouse gas
emissions and removals. We begin by reconciling terminology and mapping
ForC fields into EFDB. This process required some updates to the ForC
database structure, leading to the release of a new version of ForC
(v4.0; described here). As of May 07, 2023, ForC contained
\textasciitilde17204 independent records that would be relevant to EFDB,
1214 of which have been submitted to date. Among the data in ForC, there
is disproportionate representation of biomass (particularly aboveground)
stocks, with far fewer records for dead organic matter and soil C, and
relatively few or no records for net annual increments or C fluxes into
(gains) or out of (losses) the IPCC-defined C pools. Geographic
representation is also quite uneven, with the highest densities of
relevant records in temperate forests, and with relatively scant
representation of tropical forests in Africa and Asia. ForC represents a
diversity of stand ages, although records for young stands are primarily
limited to C stocks, as opposed to net increments of fluxes. This
distribution of records is generally reflected in the subset of records
that have been submitted to EFDB to date. In the future, forest C
estimates in EFDB can be improved through targeted research to fill
critical gaps, reporting of information required by IPCC, and continued
submission of data from scientific publications to the EFDB. Given that
climate change is rapidly impacting the world's forests, timely
reporting of recent estimates will be especially critical to accurate
forest C accounting.

\introduction[Introduction]

Forests are critical to management of atmospheric concentrations of the
greenhouse gas carbon dioxide (CO\textsubscript{2}), and thereby climate
change. In recent decades, CO\textsubscript{2} uptake by forests,
woodlands, and savannas has exceeded releases from deforestation and
other severe disturbances, resulting in a net carbon CO\textsubscript{2}
sink of \textasciitilde0.88 Gt C yr\textsuperscript{-1} \citep[all
biomes with trees,][]{xu_changes_2021} to \textasciitilde1.6 Gt C
yr\textsuperscript{-1} \citep[forests only,][]{harris_global_2021}. This
has offset an estimated 10\% to 18\% of anthropogenic
CO\textsubscript{2} emissions from fossil fuels and cement
\citep{xu_changes_2021, harris_global_2021}, dramatically slowing the
pace of atmospheric CO\textsubscript{2} accumulation and climate change.
Going into the future, the fate of this important CO\textsubscript{2}
sink is highly uncertain, depending both upon forest responses to
climate change, which are likely to reduce the sink strength
\citep{mcdowell_pervasive_2020, hammond_global_2022}, and on human
conservation, restoration, and management of forests
\citep{ipcc_climate_2019, ipcc_climate_2022}.

Reflecting their strong influence on Earth's climate, forests play a
substantial role in international plans for climate change mitigation
under the Paris Agreement \citep{unfccc_adoption_2015}. Forest
conservation, reforestation, and improved sustainable management all
have significant -- and relatively cost-effective -- potential as
climate change mitigation options, with conservation and reforestation
having the fourth and fifth largest net emission reduction potentials or
all mitigation options \citep{ipcc_summary_2022}. As of 2016,
forest-based mitigation accounted for 26\% of total planned greenhouse
gas mitigation within Nationally Determined Contributions under the
Paris Agreement \citep{grassi_key_2017}. Yet, envisioned forest-based
climate change mitigation initiatives do not always correspond to actual
emission reductions through on-the-ground implementation
\citep[e.g.,][]{badgley_systematic_2022}. One critical need for ensuring
that forest-based climate change mitigation initiatives are effective is
realistic planning and reporting, underlain by solid scientific data
\citep{anderson-teixeira_effective_2022, deng_comparing_2021}.

The International Panel on Climate Change (IPCC) provides guidance for
national greenhouse gas inventories for reporting to the United Nations
Framework Convention on Climate Change
\citep[UNFCCC,][]{ipcc_2006_2006, ipcc_2019_2019}. Under this guidance,
greenhouse gas inventories include all managed land, including most of
the world's forest land \citep{ogle_delineating_2018}. The IPCC
inventory guidelines include specific instructions for accounting for
greenhouse gas (mainly CO\textsubscript{2}) exchanges between forest
land and the atmosphere \citep{ipcc_agriculture_2006, ipcc_2019_2019}. A
tiered approach to accounting is employed, where the lowest tier (Tier
1) represents the simplest approach and relies on default parameter
values -- for example, forest carbon (C) stocks values by ecozone
\citep{fao_global_2012} and forest age class derived as the average of
published estimates \citep{ipcc_2019_2019, rozendaal_aboveground_2022}.
Tier 1 values have improved over the years as more of the relevant
underlying data has become available
\citep{requenasuarez_estimating_2019, rozendaal_aboveground_2022}, but
there remains room for continuous improvement as the science advances.
For example, the year following the release of the latest IPCC
guidelines, a more thorough analysis of C accumulation in regrowth
forests found that IPCC's Tier 1 default values underestimated C
sequestration by 32\% on average and failed to capture eight-fold
variation within ecozones \citep{cook-patton_mapping_2020}. In addition,
it was revealed that C stocks in mature African tropical montane forests
were two-thirds higher than the IPCC Tier 1 values for these forests
\citep{cuni-sanchez_high_2021}. This rapid evolution of scientific
information on C cycling in forests is valuable for informing climate
change mitigation efforts but requires improved mechanisms for
communicating the latest information from scientific researchers to the
practitioners who need reliable estimates for greenhouse gas mitigation
planning. Moreover, high variability of forest C cycling within ecozones
\citep[e.g.,][]{cook-patton_mapping_2020, cuni-sanchez_high_2021}
implies that it is useful for practitioners to have access to
locally-specific information, when available.

To improve data accessibility for C accounting, the IPCC created the
Emission Factor Database (EFDB;
\url{https://www.ipcc-nggip.iges.or.jp/EFDB/main.php}), which is
intended as a recognized library of emission factors and other
parameters that can be used for estimating greenhouse gas emissions and
removals. The EFDB can be used both for efforts to tally a nation's
intended or accomplished greenhouse gas reductions, or as a basis of
comparison for external parties to evaluate these inventories. The EFDB
encourages researchers to submit estimates of emission factors or other
related parameters from peer-reviewed journal papers or other accepted
sources for inclusion in the database. In the case of forests, emission
factors include C stocks, net increments (``stock changes''), and fluxes
(``gains'' and ``losses'') for various pools
\citep{ipcc_2006_2006, ipcc_2019_2019}.

The Global Forest Carbon Database, ForC
(\url{https://forc-db.github.io/}), is the largest collection of
published estimates of forest C stocks, increments, and annual fluxes
\citep{anderson-teixeira_forc_2018, anderson-teixeira_carbon_2021}. ForC
includes data ingested from individual publications and relevant
databases, including the Global Reforestation Opportunity Assessment
(GROA) database \citep[database doi:
10.5281/zenodo.3983644]{cook-patton_mapping_2020}, the global soil
respiration database
\citep[SRDB-V5,][]{bond-lamberty_global_2010, jian_restructured_2021}.
As of May 07, 2023, ForC contained 39855 records from 10589 plots in
1535 distinct geographical areas, along with records of stand age and
disturbance history. As such, ForC is positioned to improve forest C
accounting through the transfer of data to EFDB. The purpose of this
publication is to document that process and provide recommendations for
future improvements.

Here, we (1) review IPCC methods and definitions for forest C accounting
in the context of typical forest C estimation methodologies; (2)
describe mapping of ForC to IPCC's EFDB; (3) describe updates to ForC
(ForC v4.0), most of which were implemented to facilitate data transfer
to EFDB; (4) summarize the data in ForC that's relevant to EFDB and
records that have been transferred to date; and (5) provide
recommendations for improving data collection, analysis, database, and
accounting.

\section{IPCC methods and definitions}

The end goal of IPCC greenhouse gas inventories is to quantify
greenhouse gas emissions to, or withdrawals from, the atmosphere on an
annual basis, most commonly on a national level
\citep{ipcc_2006_2006, ipcc_2019_2019}. For each stratum of subdivision
within a land-use category, annual stock changes (\(\Delta C\); t C
yr\textsuperscript{-1}) are calculated as the sum of changes in various
pools (described in section 2.1), plus any harvested wood products. For
each pool, \(\Delta C\) may be calculated using the ``Gain-Loss
Method'', which takes the difference between gains and losses, or using
the ``Stock-Difference Method'', which computes \(\Delta C\) based on C
stocks at two points in time \citep{ipcc_2006_2006}. Thus, C cycle
variables relevant to the IPCC methodology and to EFDB include C stocks,
net annual increments, and fluxes in the IPCC-defined pools.

\subsection{Carbon pools}

Forest ecosystem C pools may be parsed in various ways, and while
certain definitions and thresholds are more common than others, there is
no single standard for measuring or reporting that is adhered to by all
-- or even most -- scientific studies. IPCC parses forest C pools into
biomass (aboveground and belowground), dead organic matter (dead wood
and litter), and soil organic matter (Table 1). While there is some
flexibility around the components included in each pool, each national
inventory must apply these in a consistent manner.

\begin{table}

\caption{\label{tab:table_pools}\textbf{IPCC-defined forest carbon pools with definitions and measurement methods.} Definitions from IPCC Table 1.1. (See Table 1.1 in IPCC guidance).}
\centering
\begin{tabu} to \linewidth {>{\raggedright}X>{\raggedright}X>{\raggedright}X>{\raggedright}X}
\hline
\textbf{pool} & \textbf{definition} & \textbf{important sources of estimate variation} & \textbf{IPCC guidance}\\
\hline
aboveground biomass & all biomass of living vegetation & minimum size censused & may exclude understory if minor component\\
\hline
 &  & include non-dicot trees? & yes\\
\hline
 &  & include dead standing? & no\\
\hline
 &  & biomass allometry & Tier 1 defaults draw on a variety of allometric models\\
\hline
belowground biomass & all biomass of live roots & all factors relevant to aboveground biomass & see above\\
\hline
 &  & allometry or assumed ratio of below- to above-ground biomass (R) & can estimate based on R\\
\hline
 &  & minimum root diameter & may exclude fine roots; suggested minimum diameter of 2 mm for fine roots\\
\hline
dead wood & all non-living woody biomass above a specified diameter, aboveground or belowground & minimum diameter & 10 cm default, but may be chosen by country\\
\hline
 &  & include belowground? & \vphantom{1} yes\\
\hline
litter & all non-living biomass smaller than dead wood but larger than soil organic matter, in various states of decomposition both above or within the mineral or organic soil & maximum diameter (= minimum diameter for deadwood) & 10 cm default, but may be chosen by country\\
\hline
 &  & minimum size (= size limit for soil organic matter) & suggested 2 mm\\
\hline
 &  & layers included & entire O horizon: litter (OL),  fumic (OF),  and  humic (OH) layers\\
\hline
 &  & include belowground? & yes\\
\hline
soil organic matter & organic carbon in mineral soils to a specified depth & sampling depth & 30 cm default, but may be chosen by country\\
\hline
\end{tabu}
\end{table}

\subsubsection{Biomass}

Biomass includes living vegetation, above- and below-ground, both woody
and herbaceous, but with a focus on woody plants and trees given their
much greater potential to sequester large amounts of C
\citep{ipcc_2006_2006}.

Aboveground biomass, which is typically \textless200 t C
ha\textsuperscript{-1} but can exceed 700 t C ha\textsuperscript{-1}
\citep{anderson-teixeira_carbon_2021}, is defined by the IPCC as ``all
biomass of living vegetation above the soil including stems, stumps,
branches, bark, seeds, and foliage''
\citep{ipcc_good_2003, ipcc_2006_2006}. IPCC's guidance is that the
understory may be excluded the understory if it constitutes a ``minor''
component, \emph{where quantitative definitions of ``understory'' and
``minor'' are not provided}, but where a commonly applied minimum size
sampling threshold for mature forests would be 10 cm stem diameter at
breast height (DBH). A recent study characterizing the contributions of
trees in different DBH classes to ecosystem C stocks and fluxes found
that trees 1 - 10 cm DBH contributed up to \textasciitilde8\%
aboveground biomass, \textasciitilde17\% aboveground woody net primary
productivity (\(ANPP_{woody.stem}\)), and \textasciitilde20\% woody
mortality (\(M_{woody}\)) of mature closed-canopy forests worldwide
\citep{piponiot_distribution_2022}. In regrowth forests, woodlands, or
savannas, small trees and shrubs contribute a much larger proportion of
C stocks and fluxes \citep{piponiot_distribution_2022, refs_from_ForC},
and, correspondingly, biomass estimates for these ecosystems tend
include smaller size classes \citep[e.g.,][]{refs_from_ForC}. While IPCC
guidance specifies that all living vegetation should be included in
biomass estimates, forest censuses and biomass estimates do not
consistently include life forms other than dicot trees (e.g., lianas,
ferns, palms, bamboo), although thhese do tend to be censused when they
consitute a large proportion of the biomass \citep{refs_from_ForC}.
Further, it is important to note that the IPCC definition of aboveground
biomass excludes standing dead wood, which is included in remote sensing
biomass estimates \citep{duncanson_aboveground_2021}.

A universal challenge in estimating biomass (living or dead) from forest
census data is applying appropriate allometric models to convert DBH
measurements to biomass. Selection of allometric models has an enormous
influence on estimates of biomass stocks, increments, of fluxes
\citep{clark_landscapescale_2000, clark_net_2001}. While trusted and
standardized allometric models are becoming increasingly available
\citep{chave_improved_2014, rejou-mechain_biomass_2017, gonzalez-akre_allodb_2022},
large uncertainties remain. IPCC Tier 1 values currently draw on studies
applying a variety of allometric models
\citep[e.g.,][]{requenasuarez_estimating_2019, rozendaal_aboveground_2022}.

Belowground biomass is defined as ``all biomass of live roots''
\citep{ipcc_good_2003, ipcc_2006_2006}, a definition including both
coarse roots, whose biomass is typically estimated based on stem
censuses and allometries or belowground to aboveground biomass ratios,
and fine roots, whose biomass is typically estimated via extraction of
roots from soil samples. The former, which is typically \textless40 t C
ha\textsuperscript{-1} \citep{anderson-teixeira_carbon_2021}, is
methodologically linked to aboveground biomass estimates, sharing the
same methodological sources of variation, but tending to be far more
uncertain \citep{ref}. Fine root biomass generally constitutes a much
smaller C pool \citep[typically \textless5 t C
ha\textsuperscript{-1},][]{anderson-teixeira_carbon_2021}, and IPCC
guidance is that it can be excluded when fine roots cannot be
distinguished empirically from soil organic matter or litter
\citep{ipcc_2006_2006}, which can be a painstaking process. Field
methods for estimating root biomass are highly variable \citep{ref}.
IPCC's default method for Tier 1 estimates is to apply a ratio of
belowground to aboveground biomass, with default factors defined based
on ecological zone, continent, and forest age
\citep{ipcc_2006_2006, ipcc_2019_2019}.

\subsubsection{Dead Organic Matter}

Dead organic matter includes all non-living biomass that is not within
the mineral soil layer and smaller than the litter size threshold. It's
inclusion in inventories is not required under Tier 1 methodology for
Forest Land remaining Forest Land (see section 2.2), but is required for
land that has transitioned to or from forest within the past 20 years
\citep{ipcc_2006_2006}.

Dead wood, which is typically \textless50 t C ha\textsuperscript{-1} but
can exceed 150 t C ha\textsuperscript{-1}
\citep{anderson-teixeira_carbon_2021}, is defined by IPCC as ``all
non-living woody biomass not contained in the litter, either standing,
lying on the ground, or in the soil''
\citep{ipcc_good_2003, ipcc_2006_2006}. This pool includes standing and
fallen dead wood, stumps, and dead roots of diameter ≥10 cm (or a
diameter specified by the country). Dead wood stocks and fluxes can be
quite variable across forests \citep{anderson-teixeira_carbon_2021}, and
can at times be the dominant pool in a forest ecosystem \citep[e.g.,
following a severe natural disturbance,][]{carmona_coarse_2002}.
However, aboveground dead wood remains relatively poorly characterized
at a global scale \citep{anderson-teixeira_carbon_2021}, and belowground
dead wood is rarely studied \citep{merganicova_dadwood_2012}. In turn,
dead wood pools are poorly characterized in large-scale forest C budgets
\citep{pan_large_2011, harris_global_2021}, and IPCC's latest Tier 1
default values are based on just 1-31 references per climate zone
\citep[Table 2.2 in][]{ipcc_2019_2019}.

Litter, which is typically \textless40 t C ha\textsuperscript{-1} but
can exceed 100 t C ha\textsuperscript{-1}
\citep{anderson-teixeira_carbon_2021}, is defined by IPCCC as including
``all non-living biomass with a diameter less than a minimum diameter
chosen by the country (for example 10 cm), lying dead, in various states
of decomposition above the mineral or organic soil''
\citep{ipcc_good_2003, ipcc_2006_2006}. As noted above, live fine roots
may be included in litter when difficult to separate empirically. The
definition includes the entire O horizion, including litter (OL), fumic
(OF), and humic (OH) layers, in addition to litter embedded within the
soil. This definition contrasts with empirical studies that focus on
aboveground litter, often including only the OL layer in the definition
of litter, and do not always specify the components included. Similar to
dead wood, litter is poorly characterized in large-scale forest C
budgets \citep{pan_large_2011, harris_global_2021}, and IPCC's latest
Tier 1 default values are based on just 1-7 references per climate zone
\citep[Table 2.2 in][]{ipcc_2019_2019}.

\subsubsection{Soil Organic Matter/ Carbon}

Soil organic matter/ carbon (SOM/ SOC), which is typically
\textgreater100 t C and can exceed 300 t C in the top two meters of soil
\citep{sanderman_soil_2017}, is defined by IPCC as ``organic carbon in
mineral and organic soils (including peat) to a specified depth chosen
by the country and applied consistently through the time series''
\citep{ipcc_good_2003, ipcc_2006_2006}. Live fine roots may be included
with soil organic matter when it is not feasible to distinguish them
empirically. The greatest source of methodological variation in
measuring SOM/ SOC is sampling depth, which has a suggested default of
30 cm but may vary by country provided that consistent criteria are
applied.

\subsection{Land classification}

IPCC defines land-use categories to include six categories -- Forest
Land, Grassland, Wetlands, Cropland, Settlements, and Other Land
\citep{ipcc_2006_2006}. Sub-divisions include land that has remained in
a particular category for \textgreater20 years (e.g., Forest Land
remaining Forest Land) and land that has been converted from one
category to another in the past 20 years (e.g., Cropland converted to
Forest Land). Forest Land is defined as at least 10-30\% crown cover of
trees with potential to reach a minimum height of 2-5 m \emph{in situ},
and shorter-stature natural vegetation would be classified as Grassland
\citep{ipcc_good_2003}. Definitions of forest are allowed to vary by
country, but must be applied consistently. Forest Land includes land
where vegetation temporarily falls below the threshold values for forest
(e.g., due to disturbance), but is expected to exceed those thresholds
in the future \citep{ipcc_good_2003}.

The UNFCCC requires greenhouse gas reporting for all managed lands in a
country, where management is defined as ``human interventions and
practices have been applied to perform production, ecological or social
functions'' \citep{ipcc_2006_2006}. This expansive definition of managed
land implies that the majority of Forest Land in most countries is
managed. However, the definition is applied differently across
countries, and the majority of governments have yet to report their
approach for defining managed land or provide maps of managed land
\citep{ogle_delineating_2018, deng_comparing_2021}.

\section{Updates to ForC (ForC v4.0)}

Previous versions of ForC
\citep{anderson-teixeira_carbon_2016, anderson-teixeira_forc_2018, anderson-teixeira_carbon_2021}
contained most of the information required by EFDB, and, more broadly,
to inform C accounting under IPCC guidelines. However, modest changes to
the structure and contents of ForC were needed in order to provide all
information required by EFDB and to improve ForC's capacity to serve as
a repository of valuable information for forest C accounting under IPCC
guidelines. To support export of data to EFDB, and to improve the
overall quality of the ForC database, we added or modified 18 fields
(Appendix A), defined 15 new variables, implemented enhanced quality
control, manually reviewed \textgreater1705 records to obtain additional
required information, and added 329 new records.

This section describes changes relative to ForC v3.0
\citep{anderson-teixeira_carbon_2021}.

\subsection{New or modified fields}

We added or modified a total of 18 fields (Appendix A). Most notably,
these included improvement of the representation of uncertainty,
recording of original units and organic matter to C conversion factors,
and expanding the information recorded in the citations table. For the
latter, we \emph{used an R script to automatically retrieve information
based on the DOI
(\href{https://github.com/forc-db/IPCC-EFDB-integration/issues/41}{issue
41})}.

\subsection{New variables}

We added a total of 15 new EFDB-relevant variables to the set of named
and defined variables (Fig. 1), counting each pair of variables with
units in C (ending in \texttt{\_C}) or organic matter (ending in
\texttt{\_OM}) as one. The majority of these were increment variables
(n=11), adding to only one previously defined increment variable
(aboveground biomass increment, \emph{delta.agb}). These are directly
related to C stocks as previously defined in ForC, with
``\emph{delta.}'' added in front of the variable name. Further, we added
variables capturing the belowground component of woody mortality
(\emph{woody.mortality\_root}) and the combined aboveground and
belowground components of woody mortality (\emph{woody.mortality}).
Although most of these variables lacked records in ForC as of May 07,
2023, their addition gave the structure such that records can be
populated over time. Finally, to provide better definition of the
previously existing variable \emph{organic.layer}, which has a nebulous
definition that reflects the varied definitions adopted by original
studies, we added two clearly defined variables: \emph{litter}
(relatively undecomposed plant material/ OL horizon), and
\emph{O.horizon} (entire O-horizon, including OL).

\begin{figure}
\includegraphics[width=14cm]{figures_tables/C_variable_mapping} \caption{\textbf{Schematic illustrating the carbon pools quantified under IPCC accounting; corresponding ForC variables, and relationships among them.} For each C pool, we show ForC variables corresponding to the stock, stock change (net annual increment), gain (influx), and loss. Most, but not all, EFDB-relevant ForC variables are shown here. Correspondence of ForC variables to IPCC criteria often depends upon measurement protocols (e.g., min DBH). Additional caveats are as follows: (a,b) branch fall and mortality of stems below census min DBH, which are necessary for a full accounting of dead organic matter production but typically assumed negligible for calculations of biomass change, are excluded by common measurement practice (a) or ForC variable definition (b); (c) assumes that leaf production equals leaf fall, or that changes in foliage biomass are negligble; (d,e) belowground components excluded by common measurement practice (d) or ForC variable definition (e); (f) excludes movement of dead wood into litter through breakage or size reduction; (g) measurements often limited to litter horizon (OL) and may exclude larger branches and stems classified as litter and/or the more decomposed layers of the O horizon.}\label{fig:fig_variable_mapping}
\end{figure}

\subsection{Quality control measures}

Prior to releasing ForC v4.0, we executed several quality control
measures. First, we implemented a system of continuous integration using
GitHub Actions \citep[\emph{sensu}][]{kim_implementing_2022} to run some
automatic checks any time the master data files are updated, including
outlier tests and checks for completeness and naming consistency of
records across data files. Second, to improve information on geographic
coordinates, we created a field to record coordinate precision (Appendix
A), and flagged and reviewed records with suspected low precision.
Third, to identify erroneous climate data, we compared ForC climate
values to those extracted from WorldClim version \textbf{\#\#}
\citep{climwin_ref} based on site coordinates. Records deviating from
WorldClim values by more variable-specific thresholds (\textgreater5°C
for mean annual temperature, \textgreater7.5°C for mean temperatures of
the warmest and coldest months, or \textgreater1 for log(mean annual
precipitation in mm)) were flagged as requiring review prior to use in
analysis or transfer to EFDB.

Because ForC v4.0 contained known duplicate records, we used R scripts
to identify likely duplicates, as detailed in
\citet{anderson-teixeira_carbon_2021}. Henceforth, we refer to the set
of records with likely duplicates removed as ``independent records''.
All records sent to EFDB were ensured to be independent and original
through manual review, as detailed below.

\subsection{Manual review of records to be sent to EFDB}

EFDB data submissions required information that was not recorded in
previous versions of ForC, but for which new fields were created for
EFDB compatibility (Appendix A). It was therefore necessary to return to
original publications to retrieve relevant information, including (1)
estimates in original units, (2) confidence intervals (when not already
in ForC), (3) whether records of interest were presented in tables or
text or digitized from figures (EFDB will not accept digitized data),
(4) whether records of interest were presented directly, as opposed to
having been calculated from related variables (for example, if a study
presents aboveground biomass and root biomass but not total biomass,
EFDB would not accept the sum of these as a valid record of total
biomass) We also checked that existing ForC records were complete and
correct.

Manual review of records was the limiting step for data transfer to
EFDB. We prioritized review of (1) records from the Forest Global Earth
Observatory
\citep[ForestGEO,][]{anderson-teixeira_ctfsforestgeo_2015, davies_forestgeo_2021},
(2) studies with confidence intervals recorded in ForC (because
uncertainty estimates are important to the IPCC), (3) original
publications containing large numbers of EFDB-relevant records, and (4)
records from tropical regions. The latter criteria was motivated by the
fact that although tropical forest is the single most important biome
for climate change mitigation \citep{refs}, ground-based data on
tropical forest C cycling tends to be more scarce due to a variety of
challenges \citep{refs, delima_making_2022}, and \emph{tropical
countries are more likely to apply Tier 1 methodology that bases forest
C budgets on previously existing data \citep{ref}.}

\subsection{Addition of new records}

In addition to reviewing existing records, we added a total of 329 new
records to ForC. These included 104 records from two studies
\citep{piponiot_distribution_2022, lutz_largediameter_2021} that were
not previously included in ForC. In addition, we created new records for
225 EFDB-relevant estimates presented in the original publication that
were not yet present in ForC.

\section{Transfer of data from ForC to EFDB}

To transfer complete, reviewed ForC records into EFDB, we created R
scripts to restructure ForC records and populate EFDB's bulk import form
(``EFDB bulk import.xlsx''). Criteria for data transfer were that (1)
records had been checked against the original study and determined to be
complete and correct, and as originally presented, (2) the original
study presented values in tables or text, as opposed to the values
having been digitized from graphs or calculated based on related
variables, and (3) the records had not previously been sent to EFDB.
Once converted into EFDB format, the records were reviewed and then sent
to the IPCC's Technical Support Unit for inclusion in EFDB. Complete
records needed to be reviewed by the EFDB editorial board prior to
posting in the database -- a process that lags behind records transfer
and had not yet been completed for all records sent as of May 07, 2023.

\subsection{Mapping ForC to EFDB}

The mapping of ForC fields into EFDB fields is summarized in Appendix B.
For the majority of fields, contents of the field in ForC was
transferred directly into an EFDB field, either as the only contents of
that field or as part of a composite record. For example, ten ForC
fields describing site location, climate, and edaphic properties all
mapped into the EFDB field \emph{Region/Regional conditions} (Appendix
B). In cases where original studies did not present 95\% confidence
intervals (required by IPCC when available) but did present information
required to calculate these (standard error or n and standard
deviation), we calculated the 95\% confidence intervals and populated
the EFDB field with this information (noting the calculation in the EFDB
field \emph{Comments from Data Provider}). For some fields, simple
conditional logic was used to populate EFDB fields based on ForC
records. For example, for stock variables presented in the original
publication in units of dry organic matter mass (as opposed to C),
several greenhouse gasses (CO\textsubscript{2}, CO, CH\textsubscript{4},
NO, NO\textsubscript{2}, N\textsubscript{2}O) were entered in the EFDB
field indicating the greenhouse gases to which the record could be
pertinent (\emph{Gases} field) because these values could be used in
calculations of greenhouse gas emissions from biomass burning
\citep{ipcc_2006_2006}; otherwise, the only pertinent greenhouse gas
would be CO\textsubscript{2}. There were two cases in which more complex
mapping was required: (1) mapping of C cycle variables (section 4.1.1)
and (2) land classification (section 4.1.2).

\subsubsection{Carbon cycle variables}

With input from the IPCC's Technical Support Unit, we reviewed the list
of ForC variables to identify those that were relevant to EFDB and to
appropriately map them into EFDB (Fig. 1). For each C pool (Table 1), we
identified variables representing organic matter or C stocks, stock
changes (a.k.a. ``net annual increments'' by IPCC, ``increments'' in
ForC), gains (a.k.a. ``gross annual increments'' by IPCC, ``fluxes'' in
ForC), and losses (``fluxes'' in ForC). As described in section 3.2, we
also defined 15 new EFDB-relevant variables that were not previously
represented in ForC. It it important to note that the correspondence of
ForC variables to IPCC criteria often depends upon measurement protocols
(``important sources of estimate variation'' in Table 1). For example,
ForC records of biomass and dead wood vary in the minimum stem diameter
censused, such that some records would match the IPCC criteria whereas
others would not. Information on minimum diameters censused and other
important sources of methodological variation are recorded as covariates
in ForC and mapped into the EFDB field \emph{Other Properties} (Appendix
B). Details on the mapping of ForC variables to EFDB -- including
associated covariates, IPCC pools (Table 1) and relevant equations
\citep{ipcc_2006_2006} -- are documented in the file
ForC\_variables\_mapping.csv in the GitHub repository associated with
this publication IPCC-EFDB-integration repository in ForC-db
organization (\url{https://github.com/forc-db/IPCC-EFDB-integration}).

\subsubsection{Land classification}

Determination of the IPCC land-use category (i.e., Forest Land,
Grassland, Wetlands, Cropland, Settlements, or Other Land; section 2.2)
was made based on the categorical ForC field \emph{dominant.life.form},
sometimes drawing upon stand age. Records with ``woody''
\emph{dominant.life.form} were classified as Forest Land. Those with
\emph{dominant.life.form} of ``woody+grass'', which in ForC is
indicative of anything from a shrub-encroached grassland to a
tree-dominated savanna, were given dual classification of Forest Land
and Grassland. This dual classification indicates that records may be
relevant to either category depending on the definition of forest
applied (varies by country). For (rare) cases where
\emph{dominant.life.form} was grass and stand age was greater than zero,
indicative of early successional vegetation, we assigned a
classification of Forest Land, consistent with the IPCC definition that
Forest Land includes land expected to succeed to forest. Cases where
\emph{dominant.life.form} was grass or crop and stand age was zero were
indicative of a control for studies of forest regrowth following
agricultural abandonment, and were classified as Grassland and Cropland,
respectively.

Classification into sub-categories was dependent upon stand age and site
history (section 2.2). For Forest Land ≥ 20 years old or of unknown
(relatively mature) age, or Forest Land \textless{} 20 years old that
was forest prior to a stand-clearing disturbance, the past land-use
category was Forest Land, making the sub-category ``Forest Land
Remaining Forest land''. For forests \textless20 years old with history
including cultivation/ tillage or grazing, past land-use categories were
Cropland and Grassland, respectively, making land-use subcategories were
``Cropland converted to Forest Land'' and ``Cropland converted to Forest
Land'', respectively. For forests \textless20 years old with unspecified
previous agricultural use, we assigned the sub-category ``Land Converted
to Forest land''. Forests \textless20 years old with unknown land use
prior to the study date were simply classified as ``Forest Land''. The
same logic was applied for savannas, but including both Forest Land and
Grassland as potentially relevant categories.

Given the lack of public information needed to determine whether lands
are classified as mangaged
\citep{ogle_delineating_2018, deng_comparing_2021}, and because the
IPCC's definition of managed land is more expansive than is commonly
applied in the scientific literature and hence in ForC, we did not
transfer any classification of land management status from ForC to the
EFDB. However, we do provide auxiliary information that should be useful
in making this determination, including geographical location and
notable disturbance events.

\section{Results}

\subsection{ForC v4.0 contents}

As of May 07, 2023, ForC (v4.0) contained 32693 independent records
(39855 total), 17204 of which were for the the 36 variables relevant to
EFDB (Fig. 1). These records were distributed across all forested
continents and ecozones (Fig. 2). The largest number of records came
from Asia, followed by North America, South America, and Europe, with
relatively few records from Africa, Australia, and Oceania (Fig. 3c).
Categorized by FAO ecozone, the greatest numbers of records came from
subtropical humid forests, temperate mountain systems, and tropical rain
forests, each with \textgreater2,000 independent records (Fig. 3b).
Boreal coniferous forests, temperate continental forests, subtropical
mountain systems, and tropical moist deciduous forests had
\textgreater1,000 independent records each, while other ecozones all had
\textless1,000 records. The most widely represented forest type was
needleleaf evergreen, followed by broadleaf deciduous and broadleaf
evergreen (Fig. 3a). In terms of stand age, the most represented age
class was 20-100 years, followed by \textless20 years and then
\textgreater100 years (Fig. 3d).

\begin{figure}
\includegraphics[width=15cm]{figures_tables/World_Map_of_sites_with_FAO_and_IPCC_data_sent} \caption{\textbf{Map of sites in ForC shaded by number of independent records relevant to (circles) and transferred to (triangles) EFDB.} Symbols are colored according to the number of records at each site. Underlying map shows FAO ecozones, which are coded as follows: Ba-Boreal coniferous forest, Bb-Boreal tundra woodland, BM-Boreal mountain systems, P-Polar, SBSh-Subtropical steppe, SBWh-Subtropical desert, SCf-Subtropical humid forest, SCs-Subtropical dry forest, SM-Subtropical mountain systems, TAr-Tropical rain forest,  TAwa-Tropical moist deciduous forest, TAwb-Tropical dry forest, TBSh-Tropical shrubland, TBWh-Tropical desert, TeBSk-Temperate steppe, TeBWk-Temperate desert, TeDc-Temperate continental forest, TeDo-Temperate oceanic forest, TeM-Temperate mountain systems, TM-Tropical mountain systems.}\label{fig:fig_map}
\end{figure}

\newpage
\begin{figure}
\includegraphics[width=15cm]{figures_tables/Histogram_n_Relevant_and_Transferred_Records} \caption{\textbf{Histograms of number of independent records in ForC relevant to (grey) and transferred to (black) EFDB, organized by (a) dominant vegetation type, (b) FAO ecozone, (c) continent, and (d) stand age.} For dominant vegetation (a), 'Other' includes deciduous needleleaf, mixed broadleaf- needleleaf, non-woody vegetation (e.g., early successional), and incompletely classified or mixed forest types. For FAO ecozones (b), codes are as listed in the caption of Figure 2.}\label{fig:fig_histograms}
\end{figure}

ForC contained records for 22 of the 36 variables relevant to EFDB
(Table 2, Fig. 1). The records were very unevenly distributed across
variables. The variable with most records was aboveground biomass,
representing 45\% of all independent records relevant to EFDB, and
aboveground biomass components (woody biomass or foliage) representing
an additional 5\%. A total of 30\% of relevant records were for root
biomass (including fine and coarse root components), while 5\% described
total biomass. The non-living pools were less represented, with 5\% of
relevant were for dead wood (including standing and fallen components),
0.4\% for litter, and 2.3\% for soil carbon.

Increment and flux variables were poorly represented (Table 2). The
increment variable with most records was the aboveground biomass
increment, representing 0.7\% of all independent records relevant to
EFDB. The only other relevant increment variable with any records was
the O horizon (litter) increment, with just 4 records. Relevant flux
variable records (n=974) were limited to the biomass pools (aboveground,
belowground, or total) and together constituted 6\% of ForC's
independent records relevant to EFDB.

\newpage
\begingroup\fontsize{8}{10}\selectfont

\begin{longtable}[t]{l|l|l|l|l}
\caption{\label{tab:table_variables}\textbf{Numbers of records of ForC variables relevant to, and sent to, EFDB.}}\\
\hline
\textbf{variable} & \textbf{n in ForC} & \textbf{n independent records in ForC} & \textbf{n reviewed} & \textbf{n sent to EFDB}\\
\hline
\endfirsthead
\caption[]{\textbf{Numbers of records of ForC variables relevant to, and sent to, EFDB.} \textit{(continued)}}\\
\hline
\textbf{variable} & \textbf{n in ForC} & \textbf{n independent records in ForC} & \textbf{n reviewed} & \textbf{n sent to EFDB}\\
\hline
\endhead
\textbf{Biomass} & \textbf{} & \textbf{} & \textbf{} & \textbf{}\\
\hline
biomass & 1095 & 847 & 93 & 48\\
\hline
delta.biomass & 0 & 0 & 0 & 0\\
\hline
NPP\_woody & 136 & 93 & 0 & 0\\
\hline
woody.mortality & 0 & 0 & 0 & 0\\
\hline
\textbf{Aboveground biomass} & \textbf{} & \textbf{} & \textbf{} & \textbf{}\\
\hline
biomass\_ag & 9050 & 7737 & 1251 & 693\\
\hline
biomass\_ag\_woody & 460 & 366 & 10 & 10\\
\hline
biomass\_ag\_foliage & 601 & 502 & 49 & 27\\
\hline
delta.agb & 166 & 128 & 123 & 123\\
\hline
ANPP\_woody & 299 & 242 & 0 & 0\\
\hline
woody.mortality\_ag & 112 & 62 & 30 & 17\\
\hline
\textbf{Belowground biomass} & \textbf{} & \textbf{} & \textbf{} & \textbf{}\\
\hline
biomass\_root & 4629 & 4180 & 123 & 55\\
\hline
biomass\_root\_fine & 931 & 594 & 18 & 18\\
\hline
biomass\_root\_coarse & 599 & 410 & 12 & 7\\
\hline
delta.biomass\_root & 0 & 0 & 0 & 0\\
\hline
delta.biomass\_root\_coarse & 0 & 0 & 0 & 0\\
\hline
delta.biomass\_root\_fine & 0 & 0 & 0 & 0\\
\hline
woody.mortality\_root & 0 & 0 & 0 & 0\\
\hline
BNPP\_root\_fine & 489 & 333 & 0 & 0\\
\hline
BNPP\_root.turnover\_fine & 91 & 56 & 0 & 0\\
\hline
BNPP\_root\_coarse & 329 & 250 & 0 & 0\\
\hline
\textbf{Dead wood} & \textbf{} & \textbf{} & \textbf{} & \textbf{}\\
\hline
deadwood & 437 & 303 & 103 & 61\\
\hline
deadwood\_standing & 152 & 120 & 17 & 17\\
\hline
deadwood\_down & 424 & 368 & 51 & 27\\
\hline
delta.deadwood & 0 & 0 & 0 & 0\\
\hline
delta.deadwood\_standing & 0 & 0 & 0 & 0\\
\hline
delta.deadwood\_down & 0 & 0 & 0 & 0\\
\hline
R\_het\_deadwood & 0 & 0 & 0 & 0\\
\hline
\textbf{Litter} & \textbf{} & \textbf{} & \textbf{} & \textbf{}\\
\hline
O.horizon & 38 & 38 & 38 & 38\\
\hline
delta.O.horizon & 4 & 4 & 4 & 4\\
\hline
litter & 30 & 30 & 23 & 23\\
\hline
delta.litter & 0 & 0 & 0 & 0\\
\hline
NPP\_litter & 0 & 0 & 0 & 0\\
\hline
R\_het\_litter & 167 & 143 & 0 & 0\\
\hline
\textbf{Soil organic matter} & \textbf{} & \textbf{} & \textbf{} & \textbf{}\\
\hline
SOM / SOC & 693 & 398 & 89 & 46\\
\hline
delta.SOM / delta.SOC & 0 & 0 & 0 & 0\\
\hline
R\_het\_soil & 0 & 0 & 0 & 0\\
\hline
\textbf{TOTAL} & \textbf{20932} & \textbf{17204} & \textbf{2034} & \textbf{1214}\\
\hline
\end{longtable}
\endgroup{}

\subsection{Data transfers to EFDB}

As of May 07, 2023, we had reviewed or added 2034 EFDB-relevant records,
1214 records of which were sent to EFDB, and 73 of which have been
reviewed, accepted, and posted (Figs. 2-3, Table 2).
\emph{{[}DETAILS{]}}

\section{Recommendations}

Based on our experience contributing forest C data to EFDB via ForC, we
make several recommendations as to how scientists can improve forest C
records in EFDB through database work (section 6.1), new data collection
and analysis (section 6.2), and reporting (section 6.3). We also
highlight notable mismatches between IPCC accounting methods and forest
C mensuration (section 6.4).

\subsection{Database needs}

There is vast potential to expand forest C data in EFDB by completing
the process of reviewing and sending data that are already in ForC
(Figs. 2-3). So far, only \textasciitilde7\% of the EFDB-relevant data
in ForC have been sent to EFDB.

Moreover, there are many published EFDB-relevant forest C data that are
not included in ForC, with more being published on a nearly daily basis.
Coverage of particular variables or regions could be vastly improved
through systematic review of the literature. \emph{(There are some
efforts underway, including a few that Susan can specify.)} Such reviews
are necessary to even develop a rigorous assessment of forest C data
that are available, versus those that require additional data collection
and analysis.

\subsection{Data collection and analysis needs}

New data collection and analysis is needed to fill notable knowledge
gaps. While aboveground biomass stocks in particular have received --
and continue to receive -- significant research attention, other pools
and variables remain poorly quantified \citep[Table
2,][]{anderson-teixeira_carbon_2021}. Furthermore, data distribution is
uneven across forest types and geographical regions (Figs. 2-3). For
instance, C cycling of tropical forests -- particularly in Africa --
remains relatively poorly characterized, in large part due to
substantial barriers to data collection and distribution
\citep{delima_making_2022} \emph{(add some more here?)}

Several variables of value for IPCC C accounting have not been
calculated and presented as would be possible given the same forest
census data and minimal extra research effort. For example, aboveground
woody mortality (\emph{woody.mortality\_ag}) and aboveground biomass
increment (\emph{delta.agb}) can be calculated from the same census data
as aboveground woody productivity (\emph{ANPP\_woody}), yet the latter
has received far more research attention, and correspondingly has far
more records in ForC \citetext{\citealp[Table
2,][]{anderson-teixeira_carbon_2021}; \citealp[but
see][]{piponiot_distribution_2022}}. Similarly, live coarse root
biomass, total biomass, and changes in both of these pools could in
theory be easily be estimated in parallel with aboveground biomass, with
the greatest barrier being availability of reliable allometries, as have
been developed for aboveground biomass
\citep{chave_improved_2014, rejou-mechain_biomass_2017, gonzalez-akre_allodb_2022}.
However, while equations for estimating root (and thereby total) biomass
require improvement, they do exist \emph{for many forest
types}\citep{refs}, and IPCC provides default recommendations of
below-ground to above-ground ratios for estimation of root biomass
\citep{ipcc_2019_2019}. In addition, standing dead trees are captured in
most forest censuses and could be used to estimate standing dead wood,
although additional data on breakage would be needed for accurate
accounting. We recommend that, when possible, researchers calculate
these, following the reporting guidelines specified in section 6.3.

\emph{Other EFDB-relevant variables require more effort but are
warranted given their importance for forest C accounting.} Given
widespread trends of increasing tree mortality \citep{refs}, including
through severe natural disturbance \citep{refs}, better characterization
of dead wood will be critical. \ldots{}

\subsection{Data reporting needs}

We recommend that, unless they have some specific reason to do
otherwise, researchers calculate and report the values according to IPCC
standards (Table 3). It is particularly noteworthy that simple decisions
on the presentation of results will determine whether the data meet the
criteria for inclusion in EFDB. Some examples are as follows: (1)
presenting data only in a figure makes it ineligible for inclusion in
EFDB, whereas presentation in a table or supplementary data file allows
inclusion; (2) direct presentation of all relevant variables allows
inclusion, whereas presenting only components of variables of interest
(e.g., parsing litter into fine woody debris, OL, OF, and OH layers) or
requiring simple mathematical operations to obtain a variable of
interest (e.g., \emph{delta.agb} = \emph{ANPP\_woody} -
\emph{woody.mortality.agb}) disqualifies data from inclusion; (3)
matching IPCC-defined thresholds for defining C pools (Table 1), which
may vary by country, can make the data far more relevant for IPCC
accounting (e.g., using a 10 cm cutoff between dead wood and litter,
presenting soil C to a depth of 30 cm). It should also be emphasized
that reporting of 95\% confidence intervals (or other metrics of error),
when applicable, is highly desirable and makes the data more relevant to
IPCC. Reports which had the most successful data transfers used EFDB
variables and had clear tables showing their results.

\begin{table}

\caption{\label{tab:table_recommendations}\textbf{Recommended best practices for reporting forest C estimates of value to national greenhouse gas inventories under IPCC guidance. } ....}
\centering
\fontsize{10}{12}\selectfont
\begin{tabu} to \linewidth {>{\raggedright}X>{\raggedright}X>{\raggedright}X}
\hline
\textbf{criteria} & \textbf{recommendation} & \textbf{rationale}\\
\hline
variables to include & When possible, calculate and present all relevant variables that can be readily caluclated based on available data. & Estimates of relevant variables are not always calcualted.\\
\hline
forest census methods & Adopt IPCC guidelines (country-specific) for minimum stem size in censues in census and reporting. Ideally, census stem down to the smallest diameter feasible. & IPCC biomass pool definition includes all living vegetation, but understory may be excluded when contribution is minor.\\
\hline
 & *taxa to include* & \\
\hline
dead organic matter sampling & *include damage estimates on standing dead trees* & \\
\hline
 & Adopt IPCC recommendations for minimum diameter of deadwood (country-specific, default 10 cm). & \\
\hline
belowground sampling & Select and report soil sampling increments to include a cutoff at 30 cm depth (or country-specific depth). & IPCC biomass pool definition includes all living vegetation\\
\hline
reporting variiables & Present each variable individually, as opposed to requiring that variables of interest be calculated from related variables. & EFDB requires that values in the database be presented in the original article, and cannot accept subsequent calculations.\\
\hline
reporting estimates & Report all relevant values in tables, text, or supplementary tables/ data files, as opposed to in figures only. & EFDB does not acceptvalues digitized from figures.\\
\hline
reporting confidence intervals & Report 95\% confidence intervals, standard error, or standard deviation and sample size. & EFDB requires confidence invervals, when possible.\\
\hline
\end{tabu}
\end{table}

For those compiling published data (e.g., for meta-analyses), the data
set can have added value if all information required by IPCC is
extracted from original publications. This includes -- but is not
limited to -- retaining original values as presented without
modification or rounding, noting whether data were digitized, recording
confidence intervals, and recording all required fields (as indicated in
the EFDB's bulk import template). The significant effort required to map
a database into EFDB has been accomplished here (Appendix B), and we
welcome other researchers to use the ForC template.

Once EFDB-relevant data are available in peer-reviewed publications,
they may be submitted directly to EFDB or may use the ForC - EFDB data
pipline developed here. For individual publications, the former option
will generally be more efficient. However, by getting the data into ForC
as well as EFDB, the latter option will allow the data to be more
broadly useful--for example, being used for basic science
\citep[e.g.,][]{banburymorgan_global_2021, anderson-teixeira_carbon_2021}
or model benchmarking \citep{fer_ecosystem_2021}.

\subsection{Mismatches between IPCC accounting methods and forest C
mensuration}

Remote sensing biomass estimates include standing dead wood
\citep{duncanson_aboveground_2021}.

IPCC accounting methods cannot leverage eddy-covariance measurements,
which are widely seen as the best available method for quantifying
ecosystem-atmosphere gas exchange.

An important challenge is that forests are changing rapidly, and data
collected a decade ago may no longer be relevant, particularly in the
cases of C increments and fluxes.

\section{Conclusions}

\clearpage

\section*{Appendix A. Updates to ForC}
\addcontentsline{toc}{section}{Appendix A. Updates to ForC}

\captionsetup[table]{labelformat=empty}

Table A1: \textbf{Table of changes to ForC fields.}
\begingroup\fontsize{8}{10}\selectfont

\begin{longtabu} to \linewidth {>{\raggedright}X>{\raggedright}X>{\raggedright}X>{\raggedright}X>{\raggedright}X}
\hline
\textbf{Table} & \textbf{Column} & \textbf{Description} & \textbf{Changes} & \textbf{Motivation}\\
\hline
\endfirsthead
\multicolumn{5}{@{}l}{\textit{(continued)}}\\
\hline
\textbf{Table} & \textbf{Column} & \textbf{Description} & \textbf{Changes} & \textbf{Motivation}\\
\hline
\endhead
Sites & coordinates.precision & Precision of geographic coordinates, as reported by source or estimated from maps. & field added & allow identification of records with poor coordinate precision\\
\hline
Measurements & data.location.within.source & Location of data within the source listed in citation.ID. & field added & facilitate review, ensure traceability\\
\hline
 & sd, se, lower95\%CI, upper 95\%CI & Standard deviation, standard error, and lower and upper 95 percent confidence intvervals, respectively. & replaces `stat` and `stat.name` & cleaner format; ability to handle assymetrical 95 percent confidence intervals\\
\hline
 & mean.in.original.units, original.units & mean value and units presented in original publication & fields added & provide IPCC with original units, reduce errors/improve reproducibility\\
\hline
 & C.conversion.factor & Assumed/ measured C content of organic matter used to convert organic matter to C. & field added & track units conversion, allow back-calculation of OM if conversion factor deemed inappropriate\\
\hline
PFT & description & Definition of the pftcode at the community level. Differs from individual level in that properly describes mixed plant functional types. & field added & \\
\hline
 & description.individual & Definition of the pftcode at the individual plant level. & field name change (previously `description`) & \\
\hline
Citations & citation.citation & Full citation. Most of these records are automatically generated in R based upon DOI lookup. & field added & field required by IPCC\\
\hline
 & citation.language & Language of original publication, automatically generated based on the title and abstract, with some manual entries and corrections. & field added & field required by IPCC\\
\hline
 & citation.url & URL of original publication, generally retrieved automatically via URL lookup. & field added & field required by IPCC\\
\hline
 & citation.abstract & Abstract, generally retrieved automatically via DOI lookup. & field added & field required by IPCC\\
\hline
 & source.type & citation source type & field added & field required by IPCC\\
\hline
 & pdf.in.repository & Indicates whether pdf of original study has been retrieved and saved in ForC's reference repository & field added & housekeeping\\
\hline
 & EFDB.ready & Indicates whether data have been checked for export to EFDB. & field added & housekeeping\\
\hline
\end{longtabu}
\endgroup{}

\clearpage

\section*{Appendix B. Mapping ForC to EFDB}
\addcontentsline{toc}{section}{Appendix B. Mapping ForC to EFDB}

Table B1: \textbf{Mapping of ForC fields to EFDB.} Details documented in
the public GitHub repository associated with the project,
IPCC-EFDB-integration repository within the ForC-db organization (file
\emph{ForC-EFDB\_mapping.csv} available at
\url{https://github.com/forc-db/IPCC-EFDB-integration/blob/main/doc/ForC-EFDB_mapping/ForC-EFDB_mapping.csv}).
See footnotes at end of table (STILL NEED TO BE PROPERLY INSERTED).
\begingroup\fontsize{8}{10}\selectfont

\begin{longtabu} to \linewidth {>{\raggedright}X>{\raggedright}X>{\raggedright}X>{\raggedright}X>{\raggedright}X}
\hline
\textbf{ForC table} & \textbf{ForC field} & \textbf{EFDB field} & \textbf{Usage} & \textbf{Required}\\
\hline
\endfirsthead
\multicolumn{5}{@{}l}{\textit{(continued)}}\\
\hline
\textbf{ForC table} & \textbf{ForC field} & \textbf{EFDB field} & \textbf{Usage} & \textbf{Required}\\
\hline
\endhead
Measurements & measurement.ID & Other Properties & direct mapping & (no)\\
\hline
 & dominant.life.form & 1996 Source/Sink Categories, 2006 Source/Sink Categories & used to determine land subcategories (see defining\_land\_subcategory.md) & yes\\
\hline
 & stand.age & 1996 Source/Sink Categories, 2006 Source/Sink Categories, Parameters/ Conditions & used to determine land subcategories (see defining\_land\_subcategory.md), directly listed in Parameters/ Conditions & (yes)\\
\hline
 & dominant.veg, veg.notes, min.dbh & Parameters/ Conditions & direct mapping/ linking to dominant.veg description & no\\
\hline
 & variable.name & - & link to variable info in ForC\_variables table & yes\\
\hline
 & date / start.date, end.date & Other Properties & direct mapping & no\\
\hline
 & mean & Value & direct mapping & yes\\
\hline
 & mean.in.original.units & Value in Common Units & direct mapping & yes\\
\hline
 & original.units & Common Unit & direct mapping & yes\\
\hline
 & lower95\%CI, upper 95\%CI, se, sd and n & Lower Confidence Limit, Upper Confidence Limit & direct or calculated & (yes)\\
\hline
 & depth, covariate\_1, cov\_1.value, covariate\_2, cov\_2.value & Other Properties & direct mapping & no\\
\hline
 & allometry\_1, allometry\_2 & Comments from Data Provider & link to biomass allometry source, when provided & no\\
\hline
 & data.location.within.source & - & confirm that data weren't digitized, facilitate finding data in original publication & yes\\
\hline
 & ForC.investigator & Data Provider, Data Provider Contact & link to Data Provider, Data Provider Contact info & yes\\
\hline
Sites & site.ID, sites.sitename & Other Properties & direct mapping & (no)\\
\hline
 & lat, lon & Region/Regional conditions & direct mapping; used to extract continent, Koeppen, and FAO.ecozone & (no)\\
\hline
 & country, state, city, masl,  mat, map & Region/Regional conditions & direct mapping & no\\
\hline
 & continent, Koeppen & Region/Regional conditions & direct mapping & auto\\
\hline
 & soil.texture, sand, silt, clay, soil.classification & Parameters/ Conditions & direct mapping & no\\
\hline
 & FAO.ecozone & Parameters/ Conditions & direct mapping & auto\\
\hline
History & date, hist.cat, hist.type & 1996 Source/Sink Categories, 2006 Source/Sink Categories, Abatement/Control technologies & used to determine distmrs.type for Source/Sink Categories, generate list of events for Abatement/Control technologies & (yes)/no**\\
\hline
 & plot.area & Other Properties & direct mapping & no\\
\hline
Plots & plot.ID, plot.name & Other Properties & direct mapping & (no)\\
\hline
 & distmrs.type & 1996 Source/Sink Categories, 2006 Source/Sink Categories & used to determine land subcategories (see defining\_land\_subcategory.md) & auto\\
\hline
 & distmrs.type, distmrs.year, regrowth.type, regrowth.year & Other Properties & direct mapping & auto\\
\hline
PFT & description & Parameters/ Conditions & direct mapping & auto\\
\hline
variables & variable.type & Gases & For stocks in unit of organic matter, gases include CO2, CO, CH4, NO, NO2, N2O. For increments, fluxes, and stocks in units of C, gases includes only CO2. & auto\\
\hline
 & variable.name & C pool, Equation & link to C pool, Equation & auto\\
\hline
 & description & Description & direct mapping & auto\\
\hline
 & extended.description & Other Properties & direct mapping & auto\\
\hline
 & units & Unit (ID) & link to IPCC units & auto\\
\hline
Citations & citation.citation & Full Technical Reference & direct mapping & yes/auto\\
\hline
 & citation.language & Reference Language & direct mapping & yes/auto\\
\hline
 & citation.url & URL & direct mapping & no/auto\\
\hline
 & citation.abstract & Abstract in English & direct mapping & no/auto\\
\hline
 & source.type & Source of Data & direct mapping & yes\\
\hline
\end{longtabu}
\endgroup{}

`Required' field indicates whether the field is required by EFDB: yes =
value required; (yes) = input required, missing value acceptable if not
reported; auto = present within ForC infrasructure, and therefore will
always be exported to EFDB ; (no) = not required for EFDB, but required
for ForC and therefore will always be exported to EFDB; no = not
required, but exported to EFDB when a value is present.

** `(yes)' for most recent severe disturbance; `no' for other history
events



\codedataavailability{use this to add a statement when having data sets
and software code
available} %% use this section when having data sets and software code available



%%%%%%%%%%%%%%%%%%%%%%%%%%%%%%%%%%%%%%%%%%
%% optional

%%%%%%%%%%%%%%%%%%%%%%%%%%%%%%%%%%%%%%%%%%

%%%%%%%%%%%%%%%%%%%%%%%%%%%%%%%%%%%%%%%%%%
\authorcontribution{(fill this in)} %% optional section

%%%%%%%%%%%%%%%%%%%%%%%%%%%%%%%%%%%%%%%%%%
\competinginterests{The authors declare no competing
interests.} %% this section is mandatory even if you declare that no competing interests are present

%%%%%%%%%%%%%%%%%%%%%%%%%%%%%%%%%%%%%%%%%%

%%%%%%%%%%%%%%%%%%%%%%%%%%%%%%%%%%%%%%%%%%
\begin{acknowledgements}
Thank you to all researchers who collected and published the data
contained in ForC, and to all research assistants and collaborators who
have helped to build the database. Funding for this study was provided
by Bezos Earth Fund to The Nature Conservancy, the Institute for Global
Environmental Strategies, WLS(?)
\end{acknowledgements}

%% REFERENCES
%% DN: pre-configured to BibTeX for rticles

%% The reference list is compiled as follows:
%%
%% \begin{thebibliography}{}
%%
%% \bibitem[AUTHOR(YEAR)]{LABEL1}
%% REFERENCE 1
%%
%% \bibitem[AUTHOR(YEAR)]{LABEL2}
%% REFERENCE 2
%%
%% \end{thebibliography}

%% Since the Copernicus LaTeX package includes the BibTeX style file copernicus.bst,
%% authors experienced with BibTeX only have to include the following two lines:
%%
\bibliographystyle{copernicus}
\bibliography{references.bib}
%%
%% URLs and DOIs can be entered in your BibTeX file as:
%%
%% URL = {http://www.xyz.org/~jones/idx_g.htm}
%% DOI = {10.5194/xyz}


%% LITERATURE CITATIONS
%%
%% command                        & example result
%% \citet{jones90}|               & Jones et al. (1990)
%% \citep{jones90}|               & (Jones et al., 1990)
%% \citep{jones90,jones93}|       & (Jones et al., 1990, 1993)
%% \citep[p.~32]{jones90}|        & (Jones et al., 1990, p.~32)
%% \citep[e.g.,][]{jones90}|      & (e.g., Jones et al., 1990)
%% \citep[e.g.,][p.~32]{jones90}| & (e.g., Jones et al., 1990, p.~32)
%% \citeauthor{jones90}|          & Jones et al.
%% \citeyear{jones90}|            & 1990


\end{document}
